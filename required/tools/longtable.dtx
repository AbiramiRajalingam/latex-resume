% \iffalse meta-comment
%
% Copyright (C) 1993-2020
%
% The LaTeX3 Project and any individual authors listed elsewhere
% in this file.
%
% This file is part of the Standard LaTeX `Tools Bundle'.
% -------------------------------------------------------
%
% It may be distributed and/or modified under the
% conditions of the LaTeX Project Public License, either version 1.3c
% of this license or (at your option) any later version.
% The latest version of this license is in
%    https://www.latex-project.org/lppl.txt
% and version 1.3c or later is part of all distributions of LaTeX
% version 2005/12/01 or later.
%
% The list of all files belonging to the LaTeX `Tools Bundle' is
% given in the file `manifest.txt'.
%
% \fi
% \iffalse
%% File: longtable.dtx Copyright (C) 1990-2020 David Carlisle
%
%<*dtx>
          \ProvidesFile{longtable.dtx}
%</dtx>
%<package>\NeedsTeXFormat{LaTeX2e}[1995/06/01]
%
%<package>\NeedsTeXFormat{LaTeX2e}[1995/05/16]
%<package>\providecommand\DeclareRelease[3]{}
%<package>\providecommand\DeclareCurrentRelease[2]{}
%<package>
%<package>\DeclareRelease{}{2020-01-02}{longtable-2020-01-07.sty}
%<package>\DeclareCurrentRelease{}{2020-02-07}
%<package>
%<package>\ProvidesPackage{longtable}
%<driver> \ProvidesFile{longtable.drv}
% \fi
%         \ProvidesFile{longtable.dtx}
          [2020-02-07 v4.14 Multi-page Table package (DPC)]
%
% \iffalse
%<*driver>
\documentclass{ltxdoc}
\usepackage{longtable}
\begin{document}
\DocInput{longtable.dtx}
\end{document}
%</driver>
% \fi
%
% \GetFileInfo{longtable.dtx}
% \title{The \textsf{longtable} package\thanks{This file
%        has version number \fileversion, last
%        revised \filedate.}}
% \author{David Carlisle\thanks{The new algorithm for aligning `chunks'
% of a table used in version 4 of this package was devised coded
% and documented by David Kastrup.}}
% \date{\filedate}
%
% \let\package\textsf
% \let\env\textsf
% \providecommand\finalclearpage{\clearpage}
%
%  \MaintainedByLaTeXTeam{tools}
%  \maketitle
%
% \begin{abstract}
%  This package defines the \env{longtable} environment, a multi-page
%  version of \env{tabular}.
% \end{abstract}
%
% \DeleteShortVerb{\|}
% \MakeShortVerb{\"}
%
% \changes{v0.00}{1989/11/06}
%    {`Version 0' distributed as longtab.sty
%      always used just one chunk for the whole table}
%
% \changes{v1.00}{1990/12/20}
%    {Initial version}
%
% \changes{v2.00}{1991/06/17}
%    {Support NFSS and array.sty}
%
% \changes{v3.00}{1992/03/16}
%    {New implementation. tables may now start anywhere on the page.}
%
% \changes{v3.01}{1992/04/06}
%    {(Michel Goossens) If a chunk ends on a line in which the first
%     entry is empty, weird errors occur. Added special begin and end
%     groups, (\cs{ifnum}0 ) stuff as  explained in Appendix D.}
%
% \changes{v3.02}{1992/04/09}
%    {(Michel Goossens) Longtable fails if the table counter is reset
%     during a document. Now use an internal counter, but still
%     increment table so \cs{caption} and \cs{label} work out right.}
%
% \changes{v3.03}{1992/06/25}
%    {Add \cs{@ifundefined\{reset@font\}} so that the documentation may
%     be processed with old versions of \LaTeX.}
% \changes{v3.03}{1992/06/25}
%    {Modify the treatment of \cs{d@llar} to match the new versions of
%     Mittelbach's array.sty (array.sty v2.0h)}
%
% \changes{v3.04}{1992/11/12}
%    {(Jean-Pierre Drucbert) Longtable failed when used with
%     \cs{includeonly}.}
% \changes{v3.04}{1992/11/12}
%    {Fix bug which stopped \cs{kill} working correctly in headings.}
% \changes{v3.04}{1992/11/12}
%    {(Graham Gough) Made \cs{setlongtables} issue a warning message.}
% \changes{v3.04}{1992/11/12}
%    {(Sebastian Rahtz) longtable ran out of memory on really long
%     tables. Another bug introduced in V3.}
%
% \changes{v3.05}{1992/11/20}
%   {(Juergen Peus) Table was hard coded into the captions, Now the
%     captions use \cs{fnum@table}, so \cs{tablename} will be used.}
%
% \changes{v3.06}{1993/01/21}
%    {(Ingo Hoffmann) longtable fails with letter style.
%     The table counter is not defined, and the .aux file is not used in
%     the same way. This version will use a .lta file for letter style.}
%
% \changes{v3.07}{1993/06/09}
%    {Allow the \LaTeX\ syntax \cs{setcounter}\{LTchunksize\}\{10\}}
%
% \changes{v3.08}{1993/06/09}
%     {Update for \LaTeXe}
%
% \changes{v3.09}{1994/03/15}
%     {New ltxdoc style}
%
% \changes{v3.11}{1994/05/22}
%      {Option handling added, new style errors and warnings}
%
% \changes{v3.12}{1994/06/30}
%      {Remove special letter class handling from v3.06, not needed for
%      new letter class}
%
% \changes{v4.00}{1996/04/08}
%      {(DK) New algorithm to align chunks devised and coded by David
%      Kastrup}
%
% \changes{v4.02}{1996/04/16}
%      {(DPC/DK) documentation improvements}
%
% \changes{v4.06}{1997/06/28}
%      {(DK) new email address}
%
%
%
% \makeatletter
% \def\@oddfoot{\normalfont\rmfamily\dotfill Page \thepage\dotfill}
% \def\@oddhead{\dotfill{\normalfont\ttfamily longtable.sty}\dotfill}
% \def\ps@titlepage{\let\@oddhead\@empty}
% \makeatother
%
%
% \setlength\LTleft\parindent
% \setlength\LTright\fill
% \setcounter{LTchunksize}{10}
%
% \def\v{\char`}
%
% ^^A \vbox to 100pt makes the page breaks the same on the first run.
% \changes{v3.08}{1993/06/09}
%     {No need to use \cs{vbox} with \LaTeXe\ minipage}
% \noindent\begin{minipage}[t][130pt]{\textwidth}
% \listoftables
% \end{minipage}
%
% \section{Introduction}
%
% The \package{longtable} package defines a new environment,
% \DescribeEnv{longtable}
% \env{longtable}, which has most of the features of the \env{tabular}
% environment, but produces tables which may be broken by \TeX's
% standard page-breaking algorithm. It also shares some features with
% the \env{table} environment. In particular it uses the same counter,
% \texttt{table}, and has a similar "\caption" command. Also, the
% standard "\listoftables" command lists tables produced by either the
% \env{table} or \env{longtable} environments.
%
% The following example uses most of the features of the \env{longtable}
% environment. An edited listing of the input for this example appears
% in Section~\ref{listing}.
%
% \textbf{Note:} Various parts of the following table will
% \textbf{not} line up correctly until this document has been run
% through \LaTeX\ several times.  This is a characteristic feature of
% this package, as described below.
%
% \begin{longtable}{@{*}r||p{1in}@{*}}
% KILLED & LINE!!!! \kill
% \caption
% [An optional table caption (used in the list of tables)]
% {A long table\label{long}}\\
% \hline\hline
% \multicolumn{2}{@{*}c@{*}}%
%      {This part appears at the top of the table}\\
% \textsc{First}&\textsc{Second}\\
% \hline\hline
% \endfirsthead
% \caption[]{(continued)}\\
% \hline\hline
% \multicolumn{2}{@{*}c@{*}}%
%       {This part appears at the top of every other page}\\
% \textbf{First}&\textbf{Second}\\
% \hline\hline
% \endhead
% \hline
% This goes at the&bottom.\\
% \hline
% \endfoot
% \hline
% These lines will&appear\\
% in place of the & usual foot\\
% at the end& of the table\\
% \hline
% \endlastfoot
% \env{longtable}  columns  are specified& in the \\
% same way as  in the \env{tabular}& environment.\\
% "@{*}r||p{1in}@{*}"& in this case.\\
% Each row ends with a& "\\" command.\\
% The "\\"  command  has an& optional\\
% argument, just as in& the\\
%  \env{tabular}&environment.\\[10pt]
% See the  effect  of "\\[10pt]"&?\\
% Lots of lines& like this.\\
% Lots of lines& like this.\\
% Lots of lines& like this.\\
% Lots of lines& like this.\\
% Also  "\hline"  may be used,&  as in \env{tabular}.\\
% \hline
% That  was a "\hline"&.\\
% \hline\hline
% That  was "\hline\hline"&.\\
% \multicolumn{2}{||c||}%
% {This is a \ttfamily\v\\multicolumn\v{2\v}\v{||c||\v}}\\
% If a  page break  occurs at a "\hline" then& a line is drawn\\
% at the bottom of one  page  and at the& top of the next.\\
% \hline
% The  "[t] [b] [c]"  argument of \env{tabular}& can  not be used.\\
% The optional argument may be  one of& "[l] [r] [c]"\\
% to specify whether  the  table  should be& adjusted\\
% to the  left, right& or centrally.\\
% \hline\hline
% Lots of lines& like this.\\
% Lots of lines& like this.\\
% Lots of lines& like this.\\
% Lots of lines& like this.\\
% Lots of lines& like this.\\
% Lots of lines& like this.\\
% Lots of lines& like this.\\
% Lots of lines& like this.\\
% Lots of lines& like this.\\
% Lots of lines& like this.\\
% Lots of lines& like this.\\
% Lots of lines& like this.\\
% Lots of lines& like this.\\
% Lots of lines& like this.\\
% Lots of lines& like this.\\
% Lots of lines& like this.\\
% Lots of lines& like this.\\
% Lots of lines& like this.\\
% Lots of lines& like this.\\
% Lots of lines& like this.\\
% Some lines may take up a lot of space, like this: &
%     \raggedleft This last column is a ``p'' column so this
%     ``row'' of the table can take up several lines. Note however that
%     \TeX\ will  never break a page within such a row. Page breaks only
%     occur between rows of the table or at "\hline" commands.
%     \tabularnewline
% Lots of lines& like this.\\
% Lots of lines& like this.\\
% Lots of lines& like this.\\
% Lots of lines& like this.\\
% Lots of lines& like this.\\
% Lots of lines& like this.\\
% Lots of lines& like this.\\
% \hline
% Lots\footnote{This is a footnote.} of lines& like this.\\
% Lots   of   lines& like this\footnote{\env{longtable} takes special
%     precautions, so that footnotes may also be used in `p' columns.}\\
% \hline
% Lots of lines& like this.\\
% Lots of lines& like this.
% \end{longtable}
%
% \section{Chunk Size}
%
% \DescribeMacro{LTchunksize}
% In order to \TeX\ multi-page tables, it is necessary to break up the
% table into smaller chunks, so that \TeX\ does  not have to keep
% everything in memory at one time. By default \env{longtable} uses 20
% rows per chunk, but this can be set by the user, with e.g.,
% "\setcounter{LTchunksize}{10}".\footnote
%    {You can also use the plain \TeX\ syntax
%    {\ttfamily\v\\LTchunksize=10.}}
% These chunks do not affect page breaking,
% thus if you are using a \TeX\ with a lot of memory, you can set
% "LTchunksize" to be several pages of the table. \TeX\ will run
% faster with a large "LTchunksize". However, if necessary,
% \env{longtable} can work with "LTchunksize" set to 1, in which case
% the memory taken up is negligible.
% Note that if you use the commands for setting the table head or foot
% (see below), the "LTchunksize" must be at least  as large as the
% number of rows in each of the head or foot sections.
%
% This document specifies "\setcounter{LTchunksize}{10}".  If you look
% at the previous table, after the \emph{first} run of \LaTeX\  you will
% see that various parts of the table do not line up.
% \LaTeX\ will also have printed a warning that the column
% widths had changed. \env{longtable} writes information onto the
% ".aux" file, so that it can line up the different chunks.
% Prior to version~4 of this package, this information was not used
% unless a "\setlongtables" command was issued,  however, now the
% information is always used, using a new algorithm\footnote{Due to
% David Kastrup.} and so "\setlongtables" is no longer needed. It is
% defined (but does nothing) for the benefit of old documents that
% use it.
%
% \begin{table}
% \centering
% \begin{tabular}{||l|l|l||}
% \hline\hline
% A&\env{tabular}& environment\\
% \hline
% within&a floating&\env{table}\\
% \hline\hline
% \end{tabular}
% \caption{A floating table}
% \end{table}
%
% \section{Captions and Headings}
%
% At the start of the table one may specify lines which are to appear at
% the top of every page (under the headline, but before the other lines
% of the table).
% \DescribeMacro{\endhead}
% The lines are entered as normal, but the last "\\" command is
% replaced by a "\endhead" command.
% \DescribeMacro{\endfirsthead}
% If the first page should have a different heading, then this should be
% entered in the same way, and terminated with the "\endfirsthead"
% command. The "LTchunksize" should be at least as large as the
% number of rows in the heading.
% \DescribeMacro{\endfoot}
% There are also "\endfoot" and "\endlastfoot"
%\DescribeMacro{\endlastfoot}
% commands which are used in the same way (at the \emph{start} of the
% table) to specify rows (or an "\hline") to appear at the bottom of
% each page. In certain situations, you may want to place lines which
% logically belong in the table body at the end of the \env{firsthead},
% or the beginning of the \env{lastfoot}. This helps to control which
% lines appear on the first and last page of the table.
%
% \DescribeMacro{\caption}%
% The "\caption{...}" command is essentially equivalent to\\
% "\multicolumn{n}{c}{\parbox{\LTcapwidth}{...}}"\\
% where \texttt{n} is the number of columns of the table. You may set
% the width of the caption with a command such as
% "\setlength{\LTcapwidth}{2in}"
% in the preamble of your document. The default is 4in. "\caption" also
% writes the information to produce an entry in the list of tables. As
% with the "\caption" command in the \env{figure} and \env{table}
% environments, an optional argument specifies the text to appear in the
% list of tables if this is different from the text to appear in the
% caption. Thus the caption for table \ref{long} was specified as
% {\ttfamily
%  "\caption"[An optional table caption
%             (used in the list of tables)]\v{A long
%  table"\label{long}"\v}}.
%
% You may wish the caption on later pages to be different to that on the
% first page. In this case put the "\caption" command in the first
% heading, and put a subsidiary caption in a "\caption[]" command in
% the main heading. If the optional argument to "\caption" is empty,
% no entry is made in the list of tables. Alternatively, if you do not
% want the table number to be printed each time, use the "\caption*"
% command.
%
% The captions are set based on the code for the \package{article}
% class.
% If you have redefined the standard "\@makecaption" command to produce
% a different format for the captions, you may
% need to make similar changes to the \package{longtable} version,
% "\LT@makecaption". See the code section for more details.
%
% A more convenient method of customising captions is given by the
% \package{caption(2)} package, which provides commands for customising
% captions, and arranges that the captions in standard environments, and
% many environments provided by packages (including \package{longtable})
% are modified in a compatible manner.
%
% You may use the "\label" command so that you can cross reference
% \env{longtable}s with "\ref". Note however, that the "\label" command
% should not be used in a heading that may appear more than once. Place
% it either in the \env{firsthead}, or in the body of the table. It
% should not be the \emph{first} command in any entry.
%
% \section{Multicolumn entries}
%
% The "\multicolumn" command may be used in \env{longtable} in exactly
% the same way as for \env{tabular}. So you may want to skip this
% section, which is rather technical, however coping with "\multicolumn"
% is one of the main problems for an environment such as
% \env{longtable}. The main effect that a user will see is that
% certain combinations of "\multicolumn" entries will result in a
% document needing more runs of \LaTeX\ before the various `chunks' of
% a table align.
%
% The examples in this section are set with "LTchunksize" set to the
% minimum value of one, to demonstrate the effects when "\multicolumn"
% entries occur in different chunks.
%
% \begin{table}[!htp]
% \begin{center}
% \LTchunksize=1
%  \makeatletter
%
%  \global\let\LT@save@row\relax
%  \let\LT@warn\@gobble
%  \let\LT@final@warn\relax
%
%  \newcommand\ltexample[1]{
%  \stepcounter{LT@tables}
%  \expandafter\let\csname LT@\romannumeral\c@LT@tables\endcsname
%                   \LT@save@row
%  \addtocounter{LT@tables}{-1}
%  \begin{longtable}{|c|c|c|}
%  \caption{A difficult {\cs{multicolumn}} combination:
%                                pass #1\label{pass#1}}\\
%    \hline
%    1&2&3\\
%    \multicolumn{3}{|c|}{wide multicolumn spanning 1--3}\\
%    \multicolumn{2}{|c|}{multicolumn 1--2}&3\\
%    wide 1&2&3\\
%    \hline
% \end{longtable}
% }
%
% \ltexample{1}
%
% \ltexample{2}
%
% \ltexample{3}
%
% \ltexample{4}
%
% \end{center}
% \end{table}
%
% Consider Table~\ref{pass1}.
% In the second chunk, \env{longtable}  sees the wide
% multicolumn entry.  At this point it thinks that the first two
% columns are very narrow. All
% the width of the multicolumn entry is assumed to be in the
% third column. (This is a `feature' of \TeX's primitive "\halign"
% command.) \env{longtable} then passes the information that there
% is a wide third column to the later chunks, with the result that the
% first pass over the table is too wide.
%
% If the `saved row' from this first pass was re-inserted into the
% table on the next pass, the table would line up in two passes, but
% would be much two wide.
%
% \DescribeMacro{\kill}%
% The solution to this problem used in Versions 1~and~2, was to use a
% "\kill" line. If a line is "\kill"ed, by using "\kill" rather than
% "\\" at the end of the line, it is used in calculating
% column widths, but removed from the final table. Thus entering
% "\kill"ed copies of the last two rows before the wide multicolumn
% entry would  mean that "\halign" `saw' the wide entries in the first
% two columns, and so would not widen the third column by so much to
% make room for the multicolumn entry.
%
% In Version~3, a new solution was introduced. If the saved row in
% the ".aux" file was not being used, \env{longtable} used a special
% `draft' form of "\multicolumn", this modified the definition, so the
% spanning entry was never considered to be wider than the columns it
% spanned. So after the first pass, the ".aux" file stored the
% widest normal entry for each column, no column was widened due to
% "\span"ned columns. By default \env{longtable} ignored the ".aux"
% file, and so each run of \LaTeX\ was considered a first pass. Once the
% "\setlongtables" declaration was given, the saved row in the ".aux"
% file, and the proper definition of "\multicolumn" were used. If any
% "\multicolumn" entry caused one of the columns to be widened, this
% information could not be passed back to earlier chunks, and so the
% table would not correctly line up until the third pass. This algorithm
% always converged in three passes as described above, but in examples
% such as the ones in Tables \ref{pass1}--\ref{pass4}, the final
% widths were not optimal as the width of column~2, which is
% determined by a "\multicolumn" entry was not known when the final
% width for column~3 was fixed, due to the fact that \emph{both}
% "\multicolumn" commands were switched from `draft' mode to `normal'
% mode at the same time.
%
% Version~4 alleviates the problem considerably.
% The first pass of the table will
% indeed have the third column much too wide. However, on the next pass
% \env{longtable} will notice the error and reduce the column width
% accordingly. If this has to propagate to chunks before the
% "\multicolumn" one, an additional pass will, of course, be
% needed. It is possible to construct tables where this rippling up of
% the correct widths takes several passes to `converge' and produce a
% table with all chunks aligned. However in
% order to need many passes one needs to construct a table with
% many overlapping "\multicolumn" entries, all being wider than the
% natural widths of the columns they span, and all occurring in
% different chunks. In the typical case the algorithm will converge
% after three or four passes, and, the benefits of not needing to edit
% the document before the final run to add "\setlongtables", and the
% better choice of final column widths in the case of multiple
% "\multicolumn" entries  will hopefully more than pay for the extra
% passes that may possibly be needed.
%
% So Table~\ref{pass1} converges after 4~passes, as seen in
% Table~\ref{pass4}.
%
% You can still speed the convergence by introducing judicious "\kill"
% lines, if you happen to have constellations like the above.
%
% If you object even to \LaTeX-ing a file twice, you should
% make the first line of
% every \env{longtable} a "\kill" line that contains the widest entry
% to be used in each column. All chunks will then line up on the first
% pass.
%
% \section{Adjustment}
%
% The optional argument of \env{longtable} controls the
% horizontal alignment of the table. The possible options are "[c]",
% "[r]" and "[l]", for  centring,
% right  and left adjustment, respectively.
% \DescribeMacro{\LTleft}
% Normally centring is the default, but this document specifies
% \DescribeMacro{\LTright}
%\begin{verbatim}
% \setlength\LTleft\parindent
% \setlength\LTright\fill
%\end{verbatim}
% in the preamble,
%  which means that the tables are set flush left, but
% indented by the usual paragraph indentation. Any lengths can be
% specified for these two parameters, but at least one of them should be
% a rubber length so that it fills up the width of the page, unless
% rubber lengths are added between the columns using the
% "\extracolsep" command.
% For instance
%\begin{verbatim}
%  \begin{tabular*}{\textwidth}{@{\extracolsep{...}}...}
%\end{verbatim}
% produces a full width table, to get a similar effect with
% \env{longtable} specify
%\begin{verbatim}
% \setlength\LTleft{0pt}
% \setlength\LTright{0pt}
% \begin{longtable}{@{\extracolsep{...}}...}
%\end{verbatim}
%
% \section{Changes}
%
% This section highlights the major changes since version~2. A more
% detailed change log may be produced at the end of the code listing
% if the "ltxdoc.cfg" file specifies
%\begin{verbatim}
%  \AtBeginDocument{\RecordChanges}
%  \AtEndDocument{\PrintChanges}
%\end{verbatim}
%
% Changes made between versions 2 and 3.
% \begin{itemize}
% \item The mechanism for adding the head and foot of the table has been
% completely rewritten. With this new mechanism, \env{longtable} does
% not need to issue a "\clearpage" at the start of the table, and so the
% table may start half way down a page. Also the "\endlastfoot" command
% which could not safely be implemented under the old scheme, has been
% added.
% \item \env{longtable} now issues an error if started in the scope of
% "\twocolumn", or the \env{multicols} environment.
% \item The separate documentation file "longtable.tex" has been
% merged with the package file, "longtable.dtx" using Mittelbach's
% \package{doc} package.
% \item Support for footnotes has been added. Note however that
% "\footnote" will not work in the `head' or `foot' sections of the
% table. In order to put a footnote in those sections (e.g., inside a
% caption), use "\footnotemark" at that point, and "\footnotetext"
% anywhere in the table \emph{body} that will fall on the same page.
% \item The treatment of "\multicolumn" has changed, making
% "\kill" lines unnecessary, at the price of sometimes requiring a
% third pass through \LaTeX.
% \item The "\newpage" command now works inside a \env{longtable}.
% \end{itemize}
%
% Changes made between versions 3 and 4.
% \begin{itemize}
% \item A new algorithm is used for aligning chunks. As well as the
% widest width in each column, \package{longtable} remembers which
% chunk produced this maximum. This allows it to check that the
% maximum is still achieved in later runs. As \package{longtable} can
% now deal with columns shrinking as the file is edited, the
% "\setlongtables" system is no longer needed and is disabled.
%
% \item An extra benefit of the new algorithm's ability to deal with
% `shrinking' columns is that it can give better (narrower) column
% widths in the case of overlapping "\multicolumn" entries in
% different chunks than the previous algorithm produced.
%
% \item The `draft' multicolumn system has been removed, along with
% related commands such as "\LTmulticolumn".
%
% \item The disadvantage of the new algorithm is that it can take more
% passes. The theoretical maximum is approximately twice the length
% of a `chain' of columns with overlapping "\multicolumn" entries,
% although in practice it usually converges as fast as the old
% version. (Which always converged in three passes once
% "\setlongtables" was  activated.)
%
% \item "\\*" and "\nopagebreak" commands may be used to control page
%  breaking.
%
% \end{itemize}
%
%
% \section{Summary}
%
% ^^A Allow the table to stick into the left margin.
% \setlength{\LTleft}{0pt plus 1fill minus 1fill}
% \setlength{\LTright}{0pt}
%
% \begin{longtable}{@{}l@{\hspace{10pt}}p{.8\linewidth}@{}}
% \caption[A summary of \env{longtable} commands]%
%         {\normalsize A summary of \env{longtable} commands}\\
% \multicolumn{2}{c}{\textbf{Parameters}}\\*
% \hline
% "\LTleft"&
%     Glue to the left of the table.           \hfill("\fill")\\
% "\LTright"&
%     Glue to the right of the table.          \hfill("\fill")\\
% "\LTpre"&
%     Glue before the table.      \hfill("\bigskipamount")\\
% "\LTpost"&
%     Glue after the table.       \hfill("\bigskipamount")\\
% "\LTcapwidth"&
%     The width of a parbox containing the caption.\hfill(4in)\\
% "LTchunksize"&
%     The number of rows per chunk.                 \hfill(20)\\[5pt]
% \multicolumn{2}{c}{\textbf{Optional
%      arguments to} \ttfamily\v\\begin\v{longtable\v}}\\*
% \hline
% \it none& Position as specified by "\LTleft" and "\LTright".\\
% "[c]"&   Centre the table.\\
% "[l]"&   Place the table flush left.\\
% "[r]"&   Place the table flush right.\\[5pt]
% \pagebreak[2]
% \multicolumn{2}{c}{\textbf{Commands
%                     to end table rows}}\\*
% \hline
% "\\"&
%     Specifies the end of a row\\
% "\\"\oarg{dim}& Ends row, then adds vertical space
%     (as in the \env{tabular} environment).\\
% "\\*"&
%     The same as "\\" but disallows a page break after the row.\\
% "\tabularnewline"&
%     Alternative to "\\" for use in the scope of "\raggedright" and
%     similar commands that redefine "\\".\\
% "\kill"&
%     Row is `killed', but is used in calculating widths.\\
% "\endhead"&
%     Specifies rows to appear at the top of every page.\\
% "\endfirsthead"&
%     Specifies rows to appear at the top the first page.\\
% "\endfoot"&
%     Specifies rows to appear at the bottom of every page.\\
% "\endlastfoot"&
%     Specifies rows to appear at the bottom of the last page.\\[5pt]
% \multicolumn{2}{c}{\textbf{\env{longtable} caption commands}}\\*
% \hline
% "\caption"\marg{caption}&
%     Caption `Table ?: \meta{caption}', and a `\meta{caption}'
%     entry in the list of tables.\\
% "\caption"\oarg{lot}\marg{caption}&
%     Caption `Table ?: \meta{caption}', and a `\meta{lot}'
%     entry in the list of tables.\\
% "\caption[]"\marg{caption}&
%     Caption `Table ?: \meta{caption}',
%     but no entry in the list of tables.\\
% "\caption*"\marg{caption}&
%     Caption `\meta{caption}', but no entry in the list of tables.\\[5pt]
% \multicolumn{2}{c}{%^^A
%        \textbf{Commands available at the start of a row}}\\*
% \hline
% "\pagebreak"&
%     Force a page break.\\*
% "\pagebreak"\oarg{val}& A `hint' between 0 and 4
%  of the desirability  of a break.\\
% "\nopagebreak"& Prohibit a page break.\\*
% "\nopagebreak"\oarg{val}& A `hint' between 0 and 4 of the undesirability
%    of a break.\\
% "\newpage"&
%     Force a page break.\\[5pt]
% \multicolumn{2}{c}{\textbf{Footnote commands
%                      available inside \env{longtable}}}\\*
% \hline
% "\footnote"&
%     Footnotes, but may not be used in the table head \& foot.\\*
% "\footnotemark"&
%     Footnotemark, may be used in the table head \& foot.\\*
% "\footnotetext"&
%     Footnote text, use in the table body.\\[5pt]
% \multicolumn{2}{c}{\textbf{Setlongtables}}\\
% \hline
% "\setlongtables"&  Obsolete command. Does nothing now.
% \end{longtable}
%
%
% \finalclearpage
% \section{Verbatim highlights from Table \protect\ref{long}}
% \label{listing}
% \begingroup\catcode`\/=0
% \begin{verbatim}
% \begin{longtable}{@{*}r||p{1in}@{*}}
% KILLED & LINE!!!! \kill
% \caption[An optional table caption /ldots]{A long table\label{long}}\\
% \hline\hline
% \multicolumn{2}{@{*}c@{*}}%
%      {This part appears at the top of the table}\\
% \textsc{First}&\textsc{Second}\\
% \hline\hline
% \endfirsthead
% \caption[]{(continued)}\\
% \hline\hline
% \multicolumn{2}{@{*}c@{*}}%
%       {This part appears at the top of every other page}\\
% \textbf{First}&\textbf{Second}\\
% \hline\hline
% \endhead
% \hline
% This goes at the&bottom.\\
% \hline
% \endfoot
% \hline
% These lines will&appear\\
% in place of the & usual foot\\
% at the end& of the table\\
% \hline
% \endlastfoot
% \env{longtable}  columns  are specified& in the \\
% same way as  in the \env{tabular}& environment.\\
% /ldots
% \multicolumn{2}{||c||}{This is a /ldots}\\
% /ldots
% Some lines may take/ldots&
%     \raggedleft This last column is a ``p'' column/ldots
%     \tabularnewline
% /ldots
% Lots of lines& like this.\\
% /ldots
% \hline
% Lots\footnote{/ldots} of lines& like this.\\
% Lots   of   lines& like this\footnote{/ldots}\\
% \hline
% Lots of lines& like this.\\
% /ldots
% \end{longtable}
% \end{verbatim}
% \endgroup
%
% \StopEventually{}
%
% \finalclearpage
% \section{The Macros}
%
%    \begin{macrocode}
%<*package>
%    \end{macrocode}
%
% \subsection{Initial code}
%
% Before declaring the package options, we must define some defaults
% here.
%
%  \begin{macro}{\LT@err}
% The error generating command
%    \begin{macrocode}
\def\LT@err{\PackageError{longtable}}
%    \end{macrocode}
%  \end{macro}
%
%  \begin{macro}{\LT@warn}
% The warning generating command
%    \begin{macrocode}
\def\LT@warn{\PackageWarning{longtable}}
%    \end{macrocode}
%  \end{macro}
%
%  \begin{macro}{\LT@final@warn}
% \changes{v4.04}{1996/05/24}
%      {Macro added}
% If any \env{longtable}s have not aligned, generate a warning at the
% end of the run at "\AtEndDocument".
%    \begin{macrocode}
\def\LT@final@warn{%
  \AtEndDocument{%
    \LT@warn{Table \@width s have changed. Rerun LaTeX.\@gobbletwo}}%
  \global\let\LT@final@warn\relax}
%    \end{macrocode}
%  \end{macro}
%
% \subsection{Options}
%
% The first two options deal with error handling. They are compatible
% with the options used by the \texttt{tracefnt} package.
%
%  \begin{macro}{errorshow}
% \emph{Only} show errors on the terminal. `warnings' are just sent to
% the log file.
% \changes{v3.14}{1995/04/25}
%      {Change string from LT to longtable}
%    \begin{macrocode}
\DeclareOption{errorshow}{%
  \def\LT@warn{\PackageInfo{longtable}}}
%    \end{macrocode}
%  \end{macro}
%
%  \begin{macro}{pausing}
% \changes{v3.14}{1995/04/25}
%      {Change string from LT to longtable}
% \changes{v3.15}{1995/06/15}
%      {Use single hash for latex/1557}
% Make every warning message into an error so \TeX\ stops.
% May be useful for debugging.
%    \begin{macrocode}
\DeclareOption{pausing}{%
  \def\LT@warn#1{%
    \LT@err{#1}{This is not really an error}}}
%    \end{macrocode}
%  \end{macro}
%
%  \begin{macro}{set}
%  \begin{macro}{final}
% \changes{v4.01}{1996/04/11}
%      {(DPC) make set and final options no op}
% The next options are just alternative syntax for the
% "\setlongtables" declaration.
%    \begin{macrocode}
\DeclareOption{set}{}
\DeclareOption{final}{}
%    \end{macrocode}
%  \end{macro}
%  \end{macro}
%
%
%    \begin{macrocode}
\ProcessOptions
%    \end{macrocode}
%
% \subsection{User Settable Parameters}
%
% \begin{macro}{\LTleft}
% \begin{macro}{\LTright}
% Glue to the left and right of the table, default "\fill" (ie
% centred).
%    \begin{macrocode}
\newskip\LTleft       \LTleft=\fill
\newskip\LTright      \LTright=\fill
%    \end{macrocode}
% \end{macro}
% \end{macro}
%
% \begin{macro}{\LTpre}
% \begin{macro}{\LTpost}
% Glue before and after the \env{longtable}.  "\bigskip" by default.
%    \begin{macrocode}
\newskip\LTpre        \LTpre=\bigskipamount
\newskip\LTpost       \LTpost=\bigskipamount
%    \end{macrocode}
% \end{macro}
% \end{macro}
%
% \begin{macro}{\LTchunksize}
% Chunk size (The number of rows taken per "\halign"). Default 20.
%    \begin{macrocode}
\newcount\LTchunksize \LTchunksize=20
%    \end{macrocode}
% \end{macro}
%
% \begin{macro}{\c@LTchunksize}
% Added in V3.07 to allow the \LaTeX\ syntax
% "\setcounter{LTchunksize}{10}".
%    \begin{macrocode}
\let\c@LTchunksize\LTchunksize
%    \end{macrocode}
% \end{macro}
%
% \begin{macro}{\LTcapwidth}
% Width of the "\parbox" containing the caption. Default 4in.
%    \begin{macrocode}
\newdimen\LTcapwidth  \LTcapwidth=4in
%    \end{macrocode}
% \end{macro}
%
% \subsection{Internal Parameters}
%
% \begin{macro}{\LT@head}
% \begin{macro}{\LT@firsthead}
% \begin{macro}{\LT@foot}
% \begin{macro}{\LT@lastfoot}
% Boxes for the table head and foot.
%    \begin{macrocode}
\newbox\LT@head
\newbox\LT@firsthead
\newbox\LT@foot
\newbox\LT@lastfoot
%    \end{macrocode}
% \end{macro}
% \end{macro}
% \end{macro}
% \end{macro}
%
% \begin{macro}{\LT@cols}
% Counter for number of columns.
%    \begin{macrocode}
\newcount\LT@cols
%    \end{macrocode}
% \end{macro}
%
% \begin{macro}{\LT@rows}
% Counter for rows up to chunksize.
%    \begin{macrocode}
\newcount\LT@rows
%    \end{macrocode}
% \end{macro}
%
% \begin{macro}{\c@LT@tables}
% Counter for the tables, added in V3.02. Previous versions just used
% the \LaTeX\ counter "table", but this fails if "table" is
% reset during a document, eg \package{report} class resets it every
% chapter.
%
% This was changed from "\newcount\LT@tables" in V3.04. \LaTeX\ counters
% are preserved correctly when "\includeonly" is used. In the rest of
% the file "\LT@tables" has been replaced by "\c@LT@tables" without
% further comment.
%    \begin{macrocode}
\newcounter{LT@tables}
%    \end{macrocode}
% \end{macro}
%
% \begin{macro}{\c@LT@chunks}
% \changes{v4.00}{1996/04/08}
%      {(DK) Chunk counter added}
% We need to count through the chunks of our tables from Version~4 on.
%    \begin{macrocode}
\newcounter{LT@chunks}[LT@tables]
%    \end{macrocode}
% \end{macro}
%
% \begin{macro}{\c@table}
% \begin{macro}{\fnum@table}
% \begin{macro}{\tablename}
% If the "table" counter is not defined (eg in "letter" style), define
% it. (Added in V3.06.)
%    \begin{macrocode}
\ifx\c@table\undefined
  \newcounter{table}
  \def\fnum@table{\tablename~\thetable}
\fi
\ifx\tablename\undefined
  \def\tablename{Table}
\fi
%    \end{macrocode}
% \end{macro}
% \end{macro}
% \end{macro}
%
% \begin{macro}{\LT@out}
% In a normal style, "longtable" uses the ".aux" file to record the
% column widths. With "letter.sty", use a separate ".lta" file.
% (Added in V3.06.)
%
% Not needed for new letter class.
%\begin{verbatim}
%\ifx\startlabels\undefined
%  \let\@auxout\@auxout
%\else
%  {\@input{\jobname.lta}}%
%  \newwrite\@auxout
%  \immediate\openout\@auxout=\jobname.lta
%\fi
%\end{verbatim}
% \end{macro}
%
% \begin{macro}{\LT@p@ftn}
% Temporary storage for footnote text in a `p' column.
%    \begin{macrocode}
\newtoks\LT@p@ftn
%    \end{macrocode}
% \end{macro}
%
% \begin{macro}{\LT@end@pen}
% Special penalty for the end of the table. Done this way to save using
% up a count register.
%    \begin{macrocode}
\mathchardef\LT@end@pen=30000
%    \end{macrocode}
% \end{macro}
%
% \subsection{The \env{longtable} environment}
%
% \begin{macro}{\longtable}
% Called by "\begin{longtable}". This implementation does not work in
% multiple column formats. "\par" added at V3.04.
%    \begin{macrocode}
\def\longtable{%
  \par
  \ifx\multicols\@undefined
  \else
     \ifnum\col@number>\@ne
       \@twocolumntrue
     \fi
  \fi
  \if@twocolumn
    \LT@err{longtable not in 1-column mode}\@ehc
  \fi
  \begingroup
%    \end{macrocode}
% Check for an optional argument.
%    \begin{macrocode}
  \@ifnextchar[\LT@array{\LT@array[x]}}
%    \end{macrocode}
% \end{macro}
%
% \begin{macro}{\LT@array}
% Start setting the alignment.
% Based on "\@array" from the \LaTeX\ kernel
% and the \package{array} package.
%
% Since Version 3.02, \package{longtable} has used the internal counter
% "\c@LT@tables". The \LaTeX\ counter "table" is still incremented
% so that "\caption" works correctly.
%    \begin{macrocode}
\def\LT@array[#1]#2{%
  \refstepcounter{table}\stepcounter{LT@tables}%
%    \end{macrocode}
% Set up the glue around the table if an optional argument given.
%    \begin{macrocode}
  \if l#1%
    \LTleft\z@ \LTright\fill
  \else\if r#1%
    \LTleft\fill \LTright\z@
  \else\if c#1%
    \LTleft\fill \LTright\fill
  \fi\fi\fi
%    \end{macrocode}
% Set up these internal commands for \env{longtable}.
% \changes{v3.13}{1994/12/08}
%      {add \cs{tabularnewline}}
% \changes{v3.14}{1995/04/25}
%      {(Mike Van Geest) rename \cs{LT@mc} to
%         \cs{LT@mcol} to allow 1100 tables}
% \changes{v4.01}{1996/04/11}
%      {(DPC) don't need multicolumn warning}
%\begin{verbatim}
%  \global\let\LT@mcw@rn\relax
%\end{verbatim}
%    \begin{macrocode}
  \let\LT@mcol\multicolumn
%    \end{macrocode}
% \changes{v3.17}{1996/01/31}
%      {Reset \cs{hline} and \cs{multicolumn} in nested tabular
%       and array, for tools/2068}
% Now redefine "\@tabarray" to restore "\hline" and "\multicolumn" so
% that arrays and tabulars nested in longtable (or in page headings on
% longtable pages) work out OK. Saving the original definitions done
% here so that you can load the \package{array} package before or after
% \package{longtable}.
%    \begin{macrocode}
  \let\LT@@tabarray\@tabarray
  \let\LT@@hl\hline
  \def\@tabarray{%
    \let\hline\LT@@hl
%    \end{macrocode}
%\begin{verbatim}
%    \let\multicolumn\LT@mcol
%\end{verbatim}
%    \begin{macrocode}
    \LT@@tabarray}%
  \let\\\LT@tabularcr\let\tabularnewline\\%
  \def\newpage{\noalign{\break}}%
%    \end{macrocode}
% \changes{v4.05}{1996/11/12}
%      {\cs{nopagebreak} and \cs{pagebreak} added}
% More or less standard definitions, but first start a "\noalign".
%    \begin{macrocode}
  \def\pagebreak{\noalign{\ifnum`}=0\fi\@testopt{\LT@no@pgbk-}4}%
  \def\nopagebreak{\noalign{\ifnum`}=0\fi\@testopt\LT@no@pgbk4}%
%    \end{macrocode}
%
%    \begin{macrocode}
  \let\hline\LT@hline \let\kill\LT@kill\let\caption\LT@caption
  \@tempdima\ht\strutbox
%    \end{macrocode}
%
% \changes{v4.08}{1998/01/20}
%      {Move \cs{@endpbox} definition earlier and define \cs{@@endpbox}
%      and \cs{@@startpbox} for non-array case. tools/2736}
%    \begin{macrocode}
  \let\@endpbox\LT@endpbox
%    \end{macrocode}
% Set up internal commands according to Lamport or Mittelbach.
%    \begin{macrocode}
  \ifx\extrarowheight\@undefined
%    \end{macrocode}
% Initialise these commands as in \env{tabular} from the \LaTeX\ kernel.
%    \begin{macrocode}
    \let\@acol\@tabacol
    \let\@classz\@tabclassz \let\@classiv\@tabclassiv
    \def\@startpbox{\vtop\LT@startpbox}%
    \let\@@startpbox\@startpbox
    \let\@@endpbox\@endpbox
    \let\LT@LL@FM@cr\@tabularcr
  \else
%    \end{macrocode}
% Initialise these commands as in \package{array}. "\d@llar"
% replaced by "\d@llarbegin" "\d@llarend"  in V3.03 to match
% \package{array} V2.0h. We do not need to set "\d@llarbegin" and
% "\d@llarend"  as the \package{array} package gives them the correct
% values at the top level.
%    \begin{macrocode}
    \advance\@tempdima\extrarowheight
    \col@sep\tabcolsep
    \let\@startpbox\LT@startpbox\let\LT@LL@FM@cr\@arraycr
  \fi
%    \end{macrocode}
% The rest of this macro is mainly based on \package{array} package, but
% should work for the standard \env{tabular} too.
%    \begin{macrocode}
  \setbox\@arstrutbox\hbox{\vrule
    \@height \arraystretch \@tempdima
    \@depth \arraystretch \dp \strutbox
    \@width \z@}%
  \let\@sharp##\let\protect\relax
%    \end{macrocode}
% Interpret the preamble argument.
%    \begin{macrocode}
   \begingroup
    \@mkpream{#2}%
%    \end{macrocode}
% We need to rename "\@preamble" here as F.M.'s scheme uses
% "\global", and we may need to nest "\@mkpream", eg for
% "\multicolumn"
% or an \env{array}.
% We do not need to worry about nested \env{longtable}s though!
%    \begin{macrocode}
    \xdef\LT@bchunk{%
%    \end{macrocode}
% \changes{v4.00}{1996/04/08}
%      {(DK) Increment Chunk counter}
%    \begin{macrocode}
       \global\advance\c@LT@chunks\@ne
       \global\LT@rows\z@\setbox\z@\vbox\bgroup
%    \end{macrocode}
% \changes{v4.05}{1996/11/12}
%      {\cs{LT@setprevdepth} added}
% The following line was added in v4.05.
% In order to get the "\penalties" to work at chunk boundaries
% Need to take more care about where and when "\lineskip" glue
% is added. The following does nothing at top of table, and in
% header chunks, but in normal body chunks it sets "\prevdepth"
% (to 0pt, but any value would do) so that "\lineskip" glue will
% be added. the important thing to note is that the glue will be
% added \emph{after} any vertical material coming from "\noalign".
%    \begin{macrocode}
       \LT@setprevdepth
%    \end{macrocode}
%
% \changes{v4.10}{2000/10/22}
%      {\cs{noexpand} added (as in array.sty) for mathtext.sty, CAR}
%    \begin{macrocode}
       \tabskip\LTleft \noexpand\halign to\hsize\bgroup
%       \tabskip\LTleft\halign to\hsize\bgroup
      \tabskip\z@ \@arstrut \@preamble \tabskip\LTright \cr}%
  \endgroup
%    \end{macrocode}
% Find out how many columns we have (store in "\LT@cols").
%    \begin{macrocode}
  \expandafter\LT@nofcols\LT@bchunk&\LT@nofcols
%    \end{macrocode}
% Get the saved row from "\LT@i"\ldots"\LT@ix" (from the
% ".aux"  file), or make a new blank row.
%    \begin{macrocode}
  \LT@make@row
%    \end{macrocode}
% A few more internal commands for \env{longtable}.
%    \begin{macrocode}
  \m@th\let\par\@empty
  \everycr{}\lineskip\z@\baselineskip\z@
%    \end{macrocode}
% Start the first chunk.
%    \begin{macrocode}
  \LT@bchunk}
%    \end{macrocode}
% \end{macro}
%
% \begin{macro}{\LT@no@pgbk}
% \changes{v4.05}{1996/11/12}
%      {Macro added}
% Can simplify the standard "\@no@pgbk" as this is vmode only
% but then need to close the "\noalign".
%    \begin{macrocode}
\def\LT@no@pgbk#1[#2]{\penalty #1\@getpen{#2}\ifnum`{=0\fi}}
%    \end{macrocode}
% \end{macro}
%
% \begin{macro}{\LT@start}
% % \changes{v4.13}{2020/01/07}
%      {fix issue https://github.com/davidcarlisle/dpctex/issues/8}
% This macro starts the process of putting the table on the current
% page. It is not called until either a "\\" or "\endlongtable" command
% ends a chunk, as we do not know until that point which of the four
% possible head or foot sections have been specified.
%
% It begins by redefining itself, so that the table is only started
% once! Until V3.04, was redefined to "\relax", now use "\endgraf" to
% force the page-breaker to wake up. The second "\endgraf" is there so
% that "\pagetotal" is updated and so takes "\LTpre" into account.
%    \begin{macrocode}
\def\LT@start{%
  \let\LT@start\endgraf
  \endgraf\penalty\z@\vskip\LTpre\endgraf
%    \end{macrocode}
% Start a new page if there is not enough room for the table head, foot,
% and one extra line.
%    \begin{macrocode}
  \dimen@\pagetotal
  \advance\dimen@ \ht\ifvoid\LT@firsthead\LT@head\else\LT@firsthead\fi
  \advance\dimen@ \dp\ifvoid\LT@firsthead\LT@head\else\LT@firsthead\fi
  \advance\dimen@ \ht\LT@foot
%    \end{macrocode}
% \changes{v3.16}{1995/11/09}
%      {Measure the first line of the table}
% At this point I used to add "\ht\@arstrutbox" and "\dp\@arstrutbox"
% as a measure of a row size. However this can fail spectacularly
% for "p" columns which might be much larger. Previous versions could
% end up with the table starting with a foot, then a page break then
% a head \emph{then} a `first head'! So now measure the first line of
% the table accurately by "\vsplit"ting it out of the first chunk.
%    \begin{macrocode}
  \dimen@ii\vfuzz
  \vfuzz\maxdimen
    \setbox\tw@\copy\z@
    \setbox\tw@\vsplit\tw@ to \ht\@arstrutbox
    \setbox\tw@\vbox{\unvbox\tw@}%
  \vfuzz\dimen@ii
  \advance\dimen@ \ht
        \ifdim\ht\@arstrutbox>\ht\tw@\@arstrutbox\else\tw@\fi
  \advance\dimen@\dp
        \ifdim\dp\@arstrutbox>\dp\tw@\@arstrutbox\else\tw@\fi
%    \end{macrocode}
%
%    \begin{macrocode}
  \advance\dimen@ -\pagegoal
  \ifdim \dimen@>\z@\vfil\break\fi
%    \end{macrocode}
% Store height of page minus table foot in  "\@colroom".
% \changes{v3.14}{1995/05/02}
%      {Set \cs{@colroom}, for tools/1584}
%    \begin{macrocode}
      \global\@colroom\@colht
%    \end{macrocode}
% If the foot is non empty, reduce the "\vsize" and "\@colroom"
% accordingly.
%    \begin{macrocode}
  \ifvoid\LT@foot\else
    \advance\vsize-\ht\LT@foot
    \global\advance\@colroom-\ht\LT@foot
    \dimen@\pagegoal\advance\dimen@-\ht\LT@foot\pagegoal\dimen@
    \maxdepth\z@
  \fi
%    \end{macrocode}
% Put the table head on the page, and then switch to the new output
% routine.
% \changes{v4.11}{2004/02/01}
%      {\cs{nobreak}, for tools/3484}
%    \begin{macrocode}
  \ifvoid\LT@firsthead\copy\LT@head\else\box\LT@firsthead\fi\nobreak
  \output{\LT@output}}
%    \end{macrocode}
% \end{macro}
%
% \begin{macro}{\endlongtable}
% Called by "\end{longtable}".
%    \begin{macrocode}
\def\endlongtable{%
%    \end{macrocode}
% Essentially add a final "\\". But as we now know the number of
% actual chunks, we first strip away all entries referring to a
% maximum entry beyond the table (this can only happen if a table has
% been shortened, or the table numbering has gone awry). In that case
% we at least start collecting valid new information with the last
% chunk of this table, by removing the width constraint.
% \changes{v4.01}{1996/04/11}
%      {(DPC) use \cs{noalign} to sneak in \cs{LT@entry@chop}}
%    \begin{macrocode}
  \crcr
  \noalign{%
    \let\LT@entry\LT@entry@chop
    \xdef\LT@save@row{\LT@save@row}}%
  \LT@echunk
  \LT@start
  \unvbox\z@
  \LT@get@widths
%    \end{macrocode}
% Write the dummy row to the ".aux" file.
% Since V3.06, use ".lta" for "letter.sty".
% \changes{v3.12}{1994/06/30}
%      {Do not write if \cs{nofiles} in operation.}
%    \begin{macrocode}
  \if@filesw
    {\let\LT@entry\LT@entry@write\immediate\write\@auxout{%
%    \end{macrocode}
% Since Version 3.02, \package{longtable} has used the internal counter
% "\c@LT@tables" rather than the \LaTeX\ counter \textsf{table}. This
% information looks entirely different from version~3
% information. Still, we don't need to rename the macro name because
% later code will consider the information to have no columns, and
% thus will throw the old data away.
%    \begin{macrocode}
      \gdef\expandafter\noexpand
        \csname LT@\romannumeral\c@LT@tables\endcsname
          {\LT@save@row}}}%
  \fi
%    \end{macrocode}
% At this point used to
% issue a warning if a "\multicolumn" has been set in draft mode.
% \changes{v4.01}{1996/04/11}
%      {(DPC) No need for multicolumn warnings}
%\begin{verbatim}
%  \LT@mcw@rn
%\end{verbatim}
% If the last chunk has different widths than the first, warn the user.
% Also trigger a warning to rerun \LaTeX\ at the end of the document.
% \changes{v4.04}{1996/05/24}
%      {Use \cs{LT@final@warn}}
%    \begin{macrocode}
  \ifx\LT@save@row\LT@@save@row
  \else
    \LT@warn{Column \@width s have changed\MessageBreak
             in table \thetable}%
    \LT@final@warn
  \fi
%    \end{macrocode}
% Force one more go with the \env{longtable} output routine.
%    \begin{macrocode}
  \endgraf\penalty -\LT@end@pen
%    \end{macrocode}
% Now close the group to return to the standard routine.
%    \begin{macrocode}
  \endgroup
%    \end{macrocode}
% Reset "\@mparbottom" to allow marginpars close to the end of the
% table.\footnote{This can not be the correct. However if it is omitted,
% there is a problem with marginpars, for example on page~3 of this
% document. Any Output Routine Gurus out there?}
%    \begin{macrocode}
  \global\@mparbottom\z@
  \pagegoal\vsize
  \endgraf\penalty\z@\addvspace\LTpost
%    \end{macrocode}
% Footnotes. As done in  the \package{multicol} package.
%    \begin{macrocode}
  \ifvoid\footins\else\insert\footins{}\fi}
%    \end{macrocode}
% \end{macro}
%
% \subsection{Counting Columns}
%
% Columns are counted by examining "\@preamble", rather than simply
% getting "\@mkpream" to increment the counter as it builds the
% preamble so that this package works with many of the packages which
% add extra column specifiers to \LaTeX's standard ones.
%
%  Version~1 counted "\@sharp"'s to calculate the number of columns,
%  this was changed for Version~2 as it does not work with the NFSS. Now
%  count "&"'s. ("lfonts.new" (and now the Standard \LaTeX\ definition)
% defines "\@tabclassz" so that "\@sharp" is inside a group.)
%
% \begin{macro}{\LT@nofcols}
% Find the next "&", then look ahead to see what is next.
%    \begin{macrocode}
\def\LT@nofcols#1&{%
  \futurelet\@let@token\LT@n@fcols}
%    \end{macrocode}
% \end{macro}
%
% \begin{macro}{\LT@n@fcols}
% Add one, then stop at an "\LT@nofcols" or look for the next
% "&". The "\expandafter" trick was added in Version~3, also the
% name changed from "\@LT@nofcols" to preserve the "\LT@" naming
% convention.
%    \begin{macrocode}
\def\LT@n@fcols{%
  \advance\LT@cols\@ne
  \ifx\@let@token\LT@nofcols
    \expandafter\@gobble
  \else
    \expandafter\LT@nofcols
  \fi}
%    \end{macrocode}
% \end{macro}
%
% \subsection{The {\ttfamily\bslash\bslash} and \cs{kill} Commands}
%
% \begin{macro}{\LT@tabularcr}
% \changes{v4.05}{1996/11/12}
%      {Code re-organised for the *-form processing.}
% The internal definition of "\\".
% In the "*" form, insert a "\nobreak" after the next "\cr" (or "\crcr").
%
% This star form processing was finally added in v4.05. For the previous
% six or seven years the comment at this point said
% \begin{quote}\small
% This definition also accepts "\\*", which acts in the same way as
% "\\". \env{tabular} does this, but \env{longtable} probably ought to
% make "\\*" prevent page breaking.
% \end{quote}
%
% "{\ifnum0=`}\fi" added in version 3.01, required if the first entry
% is empty.
% The above in fact is not good enough, as with \package{array} package
% it can introduce a "{}" group in math mode, which changes the spacing.
% So use the following variant. Added in v3.14.
% \changes{v3.14}{1995/04/25}
%      {More fun with \cs{ifnum} cf tools/1571}
%    \begin{macrocode}
\def\LT@tabularcr{%
  \relax\iffalse{\fi\ifnum0=`}\fi
  \@ifstar
    {\def\crcr{\LT@crcr\noalign{\nobreak}}\let\cr\crcr
     \LT@t@bularcr}%
    {\LT@t@bularcr}}
%    \end{macrocode}
% \end{macro}
%
% \begin{macro}{\LT@crcr}
% \changes{v4.05}{1996/11/12}
%      {Macro added}
%    \begin{macrocode}
\let\LT@crcr\crcr
%    \end{macrocode}
% \end{macro}
%
% \begin{macro}{\LT@setprevdepth}
% \changes{v4.05}{1996/11/12}
%      {Macro added}
% \changes{v4.14}{2020/02/07}
%      {spurious \cs{global} removed}
% This will be redefined to set the "\prevdepth"
%  at the start of a chunk.
%    \begin{macrocode}
\let\LT@setprevdepth\relax
%    \end{macrocode}
% \end{macro}
%
% \begin{macro}{\LT@t@bularcr}
%    \begin{macrocode}
\def\LT@t@bularcr{%
%    \end{macrocode}
% Increment the counter, and do  \env{tabular}'s "\\" or finish the
% chunk.\\ The "\expandafter" trick was added in Version~3.
% Set the "\prevdepth" at the start of a new chunk. (Done here
% so not set in header chunks).
%    \begin{macrocode}
  \global\advance\LT@rows\@ne
  \ifnum\LT@rows=\LTchunksize
    \gdef\LT@setprevdepth{%
      \prevdepth\z@
      \global\let\LT@setprevdepth\relax}%
    \expandafter\LT@xtabularcr
  \else
    \ifnum0=`{}\fi
    \expandafter\LT@LL@FM@cr
  \fi}
%    \end{macrocode}
% \end{macro}
%
% \begin{macro}{\LT@xtabularcr}
% This just looks for an optional argument.
%    \begin{macrocode}
\def\LT@xtabularcr{%
  \@ifnextchar[\LT@argtabularcr\LT@ntabularcr}
%    \end{macrocode}
% \end{macro}
%
% \begin{macro}{\LT@ntabularcr}
% The version with no optional argument.
% "\ifnum0=`{\fi}" added in version 3.01. Changed in 3.14.
%    \begin{macrocode}
\def\LT@ntabularcr{%
  \ifnum0=`{}\fi
  \LT@echunk
  \LT@start
  \unvbox\z@
  \LT@get@widths
  \LT@bchunk}
%    \end{macrocode}
% \end{macro}
%
% \begin{macro}{\LT@argtabularcr}
% The version with an optional argument.
% "\ifnum0=`{\fi}" added in version 3.01. Changed in 3.14.
%    \begin{macrocode}
\def\LT@argtabularcr[#1]{%
  \ifnum0=`{}\fi
  \ifdim #1>\z@
    \unskip\@xargarraycr{#1}%
  \else
    \@yargarraycr{#1}%
  \fi
%    \end{macrocode}
% Add the dummy row, and finish the "\halign".
%    \begin{macrocode}
  \LT@echunk
  \LT@start
  \unvbox\z@
  \LT@get@widths
  \LT@bchunk}
%    \end{macrocode}
% \end{macro}
%
% \begin{macro}{\LT@echunk}
% \changes{v4.05}{1996/11/12}
%      {\cs{unskip} added for pagebreak support}
% This ends the current chunk, and removes the dummy row.
%    \begin{macrocode}
\def\LT@echunk{%
  \crcr\LT@save@row\cr\egroup
  \global\setbox\@ne\lastbox
%    \end{macrocode}
% The following line was added in v4.05.
% \package{longtable} relies on "\lineskip" glue (which is 0pt) to
% provide break points between each row so the table may be split
%  into pages.
%
% Previous releases left the "\lineskip" glue at the end of each chunk
% that had been added when the dummy row was added. There was no glue
% at the start of the next chunk as \TeX\ normally does not put
% "\lineskip" glue at the top of a box. This meant that normally the
% chunks fitted together perfectly, however "\noalign" material at a
% chunk boundary came before the first row of the next chunk
% but after the lineskip glue at the end of this chunk. This is the
% wrong place, e.g., it means even a "\penalty10000" does not stop a
% break as the "\lineskip" glue in the previous item on the list
% provides a legal breakpoint. So now remove the "\lineskip" glue that
% was before the dummy row and introduce "\LT@setprevdepth" to set the
% "\prevdepth" at the start of the next chunk, to make sure "\lineskip"
% glue is added later.
%    \begin{macrocode}
    \unskip
  \egroup}
%    \end{macrocode}
% \end{macro}
%
% \begin{macro}{\LT@entry}
% \changes{v4.00}{1996/04/08}
%      {(DK) Macro added}
% \changes{v4.01}{1996/04/11}
%      {(DPC) Use \cs{ifhmode} trick to determine first entry}
% We here give the `basic' definition of "\LT@entry", namely that used
% in alignment templates. It has a "\kern" only if the maximum is
% imposed from a different chunk.
% The "\ifhmode" test reveals the first entry, when we don't want to add
% an "&".
%    \begin{macrocode}
\def\LT@entry#1#2{%
  \ifhmode\@firstofone{&}\fi\omit
  \ifnum#1=\c@LT@chunks
  \else
    \kern#2\relax
  \fi}
%    \end{macrocode}
% \end{macro}
%
% \begin{macro}{\LT@entry@chop}
% \changes{v4.00}{1996/04/08}
%      {(DK) Macro added}
% This definition for the argument of "\LT@save@row" is used to scrap
% all those maxima which could not be verified because they occur
% after the end of the table. This can happen only if a table has been
% shortened (or the sequencing got mixed up) since the
% previous run.
% Note that this is premature: the last chunk still is going to be
% set, and with the chopped limits.
%    \begin{macrocode}
\def\LT@entry@chop#1#2{%
  \noexpand\LT@entry
    {\ifnum#1>\c@LT@chunks
       1}{0pt%
     \else
       #1}{#2%
     \fi}}
%    \end{macrocode}
% \end{macro}
%
% \begin{macro}{\LT@entry@write}
% \changes{v4.07}{1997/10/16}
%      {Avoid use of percent. tools/2631}
% To write an entry for the "aux" file, we use a slightly
% surprising definition which has the sole purpose of avoiding
% overfull lines (which might break \TeX{}'s limits when reading the
% "aux" file, probably you'd need to have a few hundred columns before
% this happened but\ldots).
%    \begin{macrocode}
\def\LT@entry@write{%
  \noexpand\LT@entry^^J%
  \@spaces}
%    \end{macrocode}
% \end{macro}
%
% \begin{macro}{\LT@kill}
% This ends the current chunk as above, but strips off two rows, the
% `dummy row' and the `killed row' before starting the next chunk.
% Since V3.04, the old chunk is reboxed at the start of the box
% containing the next chunk. This allows "\kill" to be used in headers,
% which must be processed in a single box.
%    \begin{macrocode}
\def\LT@kill{%
  \LT@echunk
  \LT@get@widths
  \expandafter\LT@rebox\LT@bchunk}
%    \end{macrocode}
% \end{macro}
%
% \begin{macro}{\LT@rebox}
% Drop the old chunk (box0) back at the top of the new chunk, removing
% the killed row. This macro added at V3.04.
%    \begin{macrocode}
\def\LT@rebox#1\bgroup{%
  #1\bgroup
  \unvbox\z@
  \unskip
  \setbox\z@\lastbox}
%    \end{macrocode}
% \end{macro}
%
% \subsection{The Dummy Row}
%
% The dummy row is kept inside of the macro "\LT@save@row".
%
% \begin{macro}{\LT@blank@row}
% \begin{macro}{\LT@build@blank}
% \changes{v4.00}{1996/04/08}
%      {(DK) Macro added}
% \changes{v4.01}{1996/04/11}
%      {(DPC) Don't mess with \cs{multicolumn} on draft passes}
% Create a blank row if we are not using the info in the ".aux" file.
%    \begin{macrocode}
\def\LT@blank@row{%
  \xdef\LT@save@row{\expandafter\LT@build@blank
    \romannumeral\number\LT@cols 001 }}
%    \end{macrocode}
% Whoops! What's that supposed to be? A drop-in replacement for the
% first task of Appendix~D in the \TeX book. The "\romannumeral"
% produces "\LT@cols" instances of "m" followed by "i". The below
% macro then replaces the "m"s by appropriate entries.
%    \begin{macrocode}
\def\LT@build@blank#1{%
  \if#1m%
    \noexpand\LT@entry{1}{0pt}%
    \expandafter\LT@build@blank
  \fi}
%    \end{macrocode}
% \end{macro}
% \end{macro}
%
% \begin{macro}{\LT@make@row}
% \changes{v4.00}{1996/04/08}
%      {(DK) New implementation}
% \changes{v4.01}{1996/04/11}
%      {(DPC) Make this the default behaviour, not needing
%      \cs{setlongtables}}
%
% Prior to version 4, by default did not use information in the
% ".aux" file but now we can define "\LT@make@row" to use the ".aux"
% file, even on the `draft' passes.
%    \begin{macrocode}
\def\LT@make@row{%
  \global\expandafter\let\expandafter\LT@save@row
    \csname LT@\romannumeral\c@LT@tables\endcsname
  \ifx\LT@save@row\relax
    \LT@blank@row
%    \end{macrocode}
% Now a slightly difficult part comes. Before we decide making the
% template from the ".aux" file info we check that the number of
% fields has remained the same. If it hasn't, either the table format
% has changed, or we have the wrong table altogether. In both cases,
% we decide to better drop all gathered information and start over.
% \changes{v4.01}{1996/04/11}
%      {(DPC) Use \cs{if} test rather than \cs{ifx}\cs{@empty}}
%
% The expansion between "!"\ldots"!" below will be empty if the number
% of "\LT@entry" macros
% including arguments in "\LT@save@row" is equal to "\LT@cols". If it
% is not empty, we throw the row away and start from scratch.
%    \begin{macrocode}
  \else
    {\let\LT@entry\or
     \if!%
         \ifcase\expandafter\expandafter\expandafter\LT@cols
         \expandafter\@gobble\LT@save@row
         \or
         \else
           \relax
         \fi
        !%
     \else
       \aftergroup\LT@blank@row
     \fi}%
  \fi}
%    \end{macrocode}
% \end{macro}
%
% \begin{macro}{\setlongtables}
% Redefine "\LT@make@row" to use information in the ".aux" file,
% if there is a saved row for this table with the right number of
% columns.
%
% Since Version 3.02, \package{longtable} has used the internal counter
% "\c@LT@tables" rather than the \LaTeX\ counter \textsf{table}.
% The warning message was added at V3.04, as was the "\global", to stop
% save-stack overflow.
%
% Since Version 4.01 "\setlongtables" does nothing as it is not
% needed, but is defined as "\relax" for the benefit of old documents.
% \changes{v3.12}{1994/06/30}
%      {Warning altered to fit line on terminal.}
% \changes{v4.01}{1996/04/11}
%      {(DPC) make into no op}
%    \begin{macrocode}
\let\setlongtables\relax
%    \end{macrocode}
% \end{macro}
%
% \begin{macro}{\LT@get@widths}
% This is the heart of \package{longtable}. If it were not for the table
% head and foot, this macro together with the modified "\\" command
% would form the basis of quite a simple little package file for long
% tables. It is closely modelled on the "\endvrulealign" macro of
% appendix D of the \TeX book.
%    \begin{macrocode}
\def\LT@get@widths{%
%    \end{macrocode}
% "\global" added at V3.04, to stop save-stack overflow.
%    \begin{macrocode}
%    \end{macrocode}
% Loop through the last row, discarding glue, and saving box widths. At
% V3.04 changed the scratch box to 2, as the new "\kill" requires that
% "\box0" be preserved.
%    \begin{macrocode}
  \setbox\tw@\hbox{%
    \unhbox\@ne
    \let\LT@old@row\LT@save@row
    \global\let\LT@save@row\@empty
    \count@\LT@cols
    \loop
      \unskip
      \setbox\tw@\lastbox
    \ifhbox\tw@
      \LT@def@row
      \advance\count@\m@ne
    \repeat}%
%    \end{macrocode}
% Remember the widths if we are in the first chunk.
%    \begin{macrocode}
  \ifx\LT@@save@row\@undefined
    \let\LT@@save@row\LT@save@row
  \fi}
%    \end{macrocode}
% \end{macro}
%
% \begin{macro}{\LT@def@row}
% \changes{v4.00}{1996/04/08}
%      {(DK) New implementation}
% Add a column to the dummy row. Name changed from "\defLT@save@row"
% in Version~3, to preserve the "\LT@" naming convention.
%    \begin{macrocode}
\def\LT@def@row{%
%    \end{macrocode}
% We start by picking the respective entry from our old row.
% These redefinitions of "\LT@entry" are local to the group started
% in "\LT@get@widths".
%    \begin{macrocode}
  \let\LT@entry\or
  \edef\@tempa{%
    \ifcase\expandafter\count@\LT@old@row
    \else
      {1}{0pt}%
    \fi}%
%    \end{macrocode}
% Now we tack the right combination in front of "\LT@save@row":
%    \begin{macrocode}
  \let\LT@entry\relax
  \xdef\LT@save@row{%
    \LT@entry
    \expandafter\LT@max@sel\@tempa
    \LT@save@row}}
%    \end{macrocode}
% \end{macro}
%
% \begin{macro}{\LT@max@sel}
% \changes{v4.00}{1996/04/08}
%      {(DK) macro added}
% \changes{v4.01}{1996/04/11}
%      {(DPC) minor modifications}
% And this is how to select the right combination. Note that we take
% the old maximum information only if the size does not change in
% \emph{either} direction. If the size has grown, we of course have a
% new maximum. If the size has shrunk, the old maximum (which was
% explicitly not enforced because of being in the current chunk) is
% invalid, and we start with this chunk as the new size. Note that
% even in the case of equality we \emph{must} use the "\the\wd\tw@"
% construct instead of "#2" because "#2" might be read in from the
% file, and so could have "\catcode"~11 versions of "p" and~"t" in~it
% which we want to be replaced by their `proper' "\catcode"~12 versions.
%    \begin{macrocode}
\def\LT@max@sel#1#2{%
  {\ifdim#2=\wd\tw@
     #1%
   \else
     \number\c@LT@chunks
   \fi}%
  {\the\wd\tw@}}
%    \end{macrocode}
% \end{macro}
%
% \subsection{The \cs{hline} Command}
%
% \begin{macro}{\LT@hline}
% "\hline" and "\hline\hline" both produce \emph{two} lines.
% The only difference being the glue and penalties between them.
% This is so that a page break at a "\hline" produces a line on both
% pages.\footnote
% {\env{longtable} has always done this, but perhaps it would be
% better if hlines were \emph{omitted} at a page break, as the head and
% foot usually put a hline here anyway.}
% Also this "\hline" is more like a "\cline{1-\LT@cols}".
% \env{tabular}'s "\hline" would draw lines the full width of the page.
%    \begin{macrocode}
\def\LT@hline{%
  \noalign{\ifnum0=`}\fi
    \penalty\@M
    \futurelet\@let@token\LT@@hline}
%    \end{macrocode}
% \end{macro}
%
% \begin{macro}{\LT@@hline}
% This code is based on "\cline". Two copies of the line are produced,
% as described above.
%    \begin{macrocode}
\def\LT@@hline{%
  \ifx\@let@token\hline
    \global\let\@gtempa\@gobble
    \gdef\LT@sep{\penalty-\@medpenalty\vskip\doublerulesep}%
  \else
    \global\let\@gtempa\@empty
    \gdef\LT@sep{\penalty-\@lowpenalty\vskip-\arrayrulewidth}%
  \fi
  \ifnum0=`{\fi}%
  \multispan\LT@cols
     \unskip\leaders\hrule\@height\arrayrulewidth\hfill\cr
  \noalign{\LT@sep}%
  \multispan\LT@cols
     \unskip\leaders\hrule\@height\arrayrulewidth\hfill\cr
  \noalign{\penalty\@M}%
  \@gtempa}
%    \end{macrocode}
% \end{macro}
%
% \subsection{Captions}
%
% \begin{macro}{\LT@caption}
% The caption is "\multicolumn{\LT@cols}{c}{"\meta{a parbox with the
% table's caption}"}"
%    \begin{macrocode}
\def\LT@caption{%
  \noalign\bgroup
    \@ifnextchar[{\egroup\LT@c@ption\@firstofone}\LT@capti@n}
%    \end{macrocode}
% \end{macro}
%
% \begin{macro}{\LT@c@ption}
% Caption command (with [optional argument]).  "\protect" added in
% Version~3. "\fnum@table" added at V3.05.
% \changes{v3.14}{1995/05/02}
%      {Call \cs{LT@makecaption} not \cs{LT@mkcaption}}
% \changes{v3.14}{1995/05/24}
%      {Add new control argument}
%    \begin{macrocode}
\def\LT@c@ption#1[#2]#3{%
  \LT@makecaption#1\fnum@table{#3}%
  \def\@tempa{#2}%
  \ifx\@tempa\@empty\else
     {\let\\\space
     \addcontentsline{lot}{table}{\protect\numberline{\thetable}{#2}}}%
  \fi}
%    \end{macrocode}
% \end{macro}
%
% \begin{macro}{\LT@capti@n}
% Caption command (no [optional argument])
% \changes{v3.14}{1995/05/02}
%      {Call \cs{LT@makecaptionx} not \cs{LT@mkcaption}}
% \changes{v3.14}{1995/05/24}
%      {Call \cs{LT@makecaption} with \cs{@gobble} or \cs{@firstofone}}
%    \begin{macrocode}
\def\LT@capti@n{%
  \@ifstar
    {\egroup\LT@c@ption\@gobble[]}%
    {\egroup\@xdblarg{\LT@c@ption\@firstofone}}}
%    \end{macrocode}
% \end{macro}
%
%
% \begin{macro}{\LT@makecaption}
% Put the caption in a box of width 0pt, so that it never affects the
% column widths. Inside that is a "\parbox" of width
% "\LTcapwidth".
% \changes{v3.14}{1995/05/02}
%      {Renamed from \cs{LT@mkcaption},
%       and modified to call \cs{@makecaption}}
% \changes{v3.14}{1995/05/02}
%      {Use the first arg to remove counter for star form}
%    \begin{macrocode}
\def\LT@makecaption#1#2#3{%
  \LT@mcol\LT@cols c{\hbox to\z@{\hss\parbox[t]\LTcapwidth{%
%    \end{macrocode}
% Based on article class "\@makecaption", "#1" is "\@gobble" in star
% form, and "\@firstofone" otherwise.
%    \begin{macrocode}
    \sbox\@tempboxa{#1{#2: }#3}%
    \ifdim\wd\@tempboxa>\hsize
      #1{#2: }#3%
    \else
      \hbox to\hsize{\hfil\box\@tempboxa\hfil}%
    \fi
    \endgraf\vskip\baselineskip}%
  \hss}}}
%    \end{macrocode}
% \end{macro}
%
%
% \subsection{The Output Routine}
%
% The method used here for interfacing a special purpose output routine
% to the standard \LaTeX\ routine is lifted straight out of
% F.~Mittelbach's \package{multicol} package.
%
% \begin{macro}{\LT@output}
%     Actually this is not so bad, with FM leading the way.
%    \begin{macrocode}
\def\LT@output{%
  \ifnum\outputpenalty <-\@Mi
    \ifnum\outputpenalty > -\LT@end@pen
%    \end{macrocode}
%     If this was a float or a marginpar we complain.
%    \begin{macrocode}
      \LT@err{floats and marginpars not allowed in a longtable}\@ehc
    \else
%    \end{macrocode}
% We have reached the end of the table, on the scroll at least,
%    \begin{macrocode}
      \setbox\z@\vbox{\unvbox\@cclv}%
      \ifdim \ht\LT@lastfoot>\ht\LT@foot
%    \end{macrocode}
% The last foot might not fit, so:\footnote{An alternative would be to
% vsplit off a bit of the last chunk, so that the last page did not just
% have head and foot sections, but it is hard to do this in a consistent
% manner.}
%    \begin{macrocode}
        \dimen@\pagegoal
        \advance\dimen@-\ht\LT@lastfoot
        \ifdim\dimen@<\ht\z@
          \setbox\@cclv\vbox{\unvbox\z@\copy\LT@foot\vss}%
          \@makecol
          \@outputpage
          \setbox\z@\vbox{\box\LT@head}%
%    \end{macrocode}
% End of "\ifdim\dimen@<\ht\@cclc".
%    \begin{macrocode}
        \fi
%    \end{macrocode}
% End of "\ifdim \ht\LT@lastfoot > \ht\LT@foot".
%    \begin{macrocode}
      \fi
%    \end{macrocode}
% Reset "\@colroom".
% \changes{v3.14}{1995/05/02}
%      {Reset \cs{@colroom}, for tools/1584}
%    \begin{macrocode}
      \global\@colroom\@colht
      \global\vsize\@colht
%    \end{macrocode}
% Put the last page of the table on to the main vertical list.
%    \begin{macrocode}
      \vbox
        {\unvbox\z@\box\ifvoid\LT@lastfoot\LT@foot\else\LT@lastfoot\fi}%
%    \end{macrocode}
% End of "\ifnum\outputpenalty > -\LT@end@pen".
%    \begin{macrocode}
    \fi
%    \end{macrocode}
% Else "\outputpenalty > -\@Mi".
%    \begin{macrocode}
  \else
%    \end{macrocode}
% If we have not reached the end of the table,
%    \begin{macrocode}
    \setbox\@cclv\vbox{\unvbox\@cclv\copy\LT@foot\vss}%
    \@makecol
    \@outputpage
%    \end{macrocode}
% Reset "\vsize".
% \changes{v3.14}{1995/05/02}
%      {Reset \cs{vsize}, for tools/1584}
%    \begin{macrocode}
      \global\vsize\@colroom
%    \end{macrocode}
% Put the head at the top of the next page.
% \changes{v4.11}{2004/02/01}
%      {\cs{nobreak}, for tools/3484}
%    \begin{macrocode}
    \copy\LT@head\nobreak
%    \end{macrocode}
% End of "\ifnum\outputpenalty <-\@Mi".
%    \begin{macrocode}
  \fi}
%    \end{macrocode}
% \end{macro}
%
% \subsection{Commands for the table head and foot}
%
% \begin{macro}{\LT@end@hd@ft}
% The core of "\endhead" and friends. Store the current chunk in the
% box specified by "#1". Issue an error if the table has already
% started. Then start a new chunk.
%    \begin{macrocode}
\def\LT@end@hd@ft#1{%
  \LT@echunk
%    \end{macrocode}
% Changed from "\relax" to "\endgraf" at V3.04, see "\LT@start".
%    \begin{macrocode}
  \ifx\LT@start\endgraf
    \LT@err
     {Longtable head or foot not at start of table}%
     {Increase LTchunksize}%
  \fi
  \setbox#1\box\z@
  \LT@get@widths
  \LT@bchunk}
%    \end{macrocode}
% \end{macro}
%
% \begin{macro}{\endfirsthead}
% \begin{macro}{\endhead}
% \begin{macro}{\endfoot}
% \begin{macro}{\endlastfoot}
% Call "\LT@end@hd@ft" with the appropriate box.
%    \begin{macrocode}
\def\endfirsthead{\LT@end@hd@ft\LT@firsthead}
\def\endhead{\LT@end@hd@ft\LT@head}
\def\endfoot{\LT@end@hd@ft\LT@foot}
\def\endlastfoot{\LT@end@hd@ft\LT@lastfoot}
%    \end{macrocode}
% \end{macro}
% \end{macro}
% \end{macro}
% \end{macro}
%
% \subsection{The \cs{multicolumn} command}
%
% Earlier versions needed a special `draft' form of "\multicolumn".
% This is not needed in version~4, and so these commands have been
% removed.
%
% \begin{macro}{\LTmulticolumn}
% \changes{v4.01}{1996/04/11}
%      {(DPC) macro removed}
% \end{macro}
%
% \begin{macro}{\LT@mcwarn}
% \changes{v4.01}{1996/04/11}
%      {(DPC) macro removed}
% \end{macro}
%
% \subsection{Footnotes}
%
% The standard "\footnote" command works in a "c" column, but we
% need to modify the definition in a "p" column to overcome the
% extra level of boxing. These macros are based on the \package{array}
% package, but should be OK for the standard \env{tabular}.
%
% \begin{macro}{\LT@startpbox}
% \changes{v4.09}{1998/05/13}
%      {Use \cs{setlength}, so that calc extensions apply. tools/2793}
% Add extra code to switch the definition of "\@footnotetext".
%    \begin{macrocode}
\def\LT@startpbox#1{%
  \bgroup
%    \end{macrocode}
% \changes{v4.12}{2019/02/06}{Color group to match array package}
%    \begin{macrocode}
    \color@begingroup
    \let\@footnotetext\LT@p@ftntext
    \setlength\hsize{#1}%
    \@arrayparboxrestore
    \vrule \@height \ht\@arstrutbox \@width \z@}
%    \end{macrocode}
% \end{macro}
%
% \begin{macro}{\LT@endpbox}
% After the parbox is closed, expand "\LT@p@ftn" which will execute a
% series of\\
% "\footnotetext["\meta{num}"]{"\meta{note}"}"\\
% commands. After being lifted out of the parbox, they can migrate on
% their own from here.
% \changes{v3.10}{1994/05/15}
%     {Use \cs{@finalstrut}}
% \changes{v4.03}{1996/05/07}
%     {Use \cs{hfil} for tools/2120}
%    \begin{macrocode}
\def\LT@endpbox{%
  \@finalstrut\@arstrutbox
%    \end{macrocode}
% \changes{v4.12}{2019/02/06}{Color group to match array package}
%    \begin{macrocode}
  \color@endgroup
  \egroup
  \the\LT@p@ftn
  \global\LT@p@ftn{}%
  \hfil}
%    \end{macrocode}
% \end{macro}
%
% \begin{macro}{\LT@p@ftntext}
% Inside the `p' column, just save up the footnote text in a token
% register.
%    \begin{macrocode}
\def\LT@p@ftntext#1{%
  \edef\@tempa{\the\LT@p@ftn\noexpand\footnotetext[\the\c@footnote]}%
  \global\LT@p@ftn\expandafter{\@tempa{#1}}}%
%    \end{macrocode}
% \end{macro}
%
%    \begin{macrocode}
%</package>
%    \end{macrocode}
%
% \Finale
\endinput
