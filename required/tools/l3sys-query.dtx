% \iffalse
%% Source File: l3sys-query.dtx
%% Copyright (C) 2024
%%
%% The LaTeX Project and any individual authors listed elsewhere
%% in this file.
%%
%% This file may be distributed under the terms of the LPPL.
%% See README for details.
%
%<*dtx>
          \ProvidesFile{l3sys-query.dtx}
%</dtx>
%<package>\ProvidesPackage{l3sys-query}
%<driver> \ProvidesFile{l3sys-query.drv}
% \fi
%         \ProvidesFile{l3sys-query.dtx}
       [2024-03-13 v1.0a LaTeX2e interface for l3sys file queries]
%
% \iffalse
%<*driver>
\documentclass{l3doc}%{ltxdoc}
\begin{document}
\DocInput{l3sys-query.dtx}
\end{document}
%</driver>
% \fi
%
% \GetFileInfo{l3sys-query.dtx}
%
% \title{\LaTeX\ document interface to the \pkg{l3sys-query} script:
% System queries for LaTeX using Lua\thanks{This file
%        has version number \fileversion, last
%        revised \filedate.}}
% \author{\LaTeX\ project}
% \date{\filedate}
%
%
% \maketitle
%
% \tableofcontents
%
% \section{Introduction}
% 
% \TeX{} engines provide only very limited access to information about the system
% they are used on: using primitives, one can for example get the size of a single
% file, but not a list of files in a given location. For most documents, this is
% not an issue as they are self-contained. However, for cases where
% \enquote{dynamic} construction of parts of a document is needed based on file
% lists or other system-dependent data, methods to obtain this from (restricted)
% shell escape are desirable.
% 
% Security considerations mean that directly querying the system shell is
% problematic for general use. Instead, \emph{restricted} shell escape may be used
% to get many details, provided a suitable tool is available to provide the
% information in a platform-neutral and security-conscious way. The Java program
% \texttt{texosquery}, written by Nicola Talbot, has been available for a number
% of years to provide this facility. As well as file system insight,
% \texttt{texosquery} also provides for example locale data and other system
% information. However, the requirement for Java means that the script is not
% automatically usable when a \TeX{} system is installed.
% 
% The \LaTeX{} team have therefore provided a Lua-based script,
% \texttt{l3sys-query}, which conforms to the security requirements of \TeX{} Live
% using Lua to obtain the system information. This means that it can be used
% \enquote{out of the box} across platforms. The facilities provided by
% \texttt{l3sys-query} are more limited than \texttt{texosquery}, partly as some
% information is available in modern \TeX{} systems using primitives, and partly
% as the aim of \texttt{l3sys-query} is to provide information where there are
% defined use cases. Requests for additional data interfaces are welcome.
%
% This package provides a document level \textit{Key-Value} interface
% to The \texttt{l3sys-query} functionality, although as a convenience
% it also summarizes the documentation for the L3 programming
% interface layer (prvided by the \pkg{l3sys} module of
% \pkg{l3kernel}, and of the command line Lua script
% \texttt{l3sys-query} which is distributed separately and provides
% the underlying functionality.
%
% 
% 
% \section{The document level key-value  interface\label{sec:kv}}
%
% This provides an interface to \texttt{l3sys-query} based on
% \pkg{l3keys} that allows the options to be specified and checked
% individually. Internally the supplied key values are used to build
% up the arguments to the commands described above in the L3
% programming layer commands described in section~\ref{sec:expl3}
%
% Results containing path separators \emph{always}
% use~|/|, irrespective of the platform in use.
% 
%
% \begin{function}
%   {\QueryWorkingDirectory}
%   \begin{syntax}
%     \cs{QueryWorkingDirectory} \Arg{result cmd}
%   \end{syntax}
%   Defines supplied command \meta{result cmd} to hold
%   the absolute path to the current working directory of the
%   \TeX\ system.
%   This is the directory (folder) from which \TeX\ was started, so usually
%   the location of the main document file.
% \end{function}
% 
% \begin{function}
%   {\QueryFiles, \QueryFilesTF}
%   \begin{syntax}
%     \cs{QueryFiles} \oarg{options} \Arg{spec} \Arg{function}
%     \cs{QueryFilesTF} \oarg{options} \Arg{spec} \Arg{function} \Arg{pre code} \Arg{empty list code}
%   \end{syntax}
%   This generates a file list  based on the \meta{spec} and \meta{options}.
%   The command then applies \meta{function}  to each item in the sequence of filenames.
%   The \meta{function} should be a macro body which will be passed the file path as |#1|.
% 
%  The TF version executes the T (\meta{pre code}) argument before iterating over the list
%  and the F (\meta{empty list code}) argument if the list is empty.
% \end{function}
% 
%  Note that this interface in mapping directly over the sequence of
%  filenames does not allow some uses which are provided by the
%  programming interface described in the following section, which allows
%  the sequence to be manipulated before being used.
% 
%  The defined keys map vey closely to the options of the
%  \texttt{l3sys-query} command which is described in
%  section~\ref{sec:lua}.
% 
%  \begin{itemize}
%  \item 
%  The keys \texttt{recursive}, \texttt{ignore-case}, \texttt{reverse}, \texttt{pattern}
%  take no values and map directly to the command line options of the same name.
% 
%  \item
%  The key \texttt{sort} accepts the values \texttt{date} and \texttt{name}.
% 
%  \item
%  The key \texttt{type} accepts \texttt{d} or \texttt{f}.
% 
%  \item
%  The key \texttt{exclude} accepts a glob (or Lua pattern) mathing
%  files to be excluded.  The package arranges that the quoting of the
%  argument is automatically added if unrestricted shell escape is
%  enabled.
% 
% \end{itemize}
% 
%  Within the main \meta{spec} argument, and the value of the \texttt{exclude} key,
%  the following characters may need special handling.
%  |~| may be used (which \texttt{kpathsea} uses to denote the users home directory).
%  |\%|, or at the top level |%|, may be used to produce a literal |%|
%  which may be especially useful if the \texttt{pattern} key is used
%  as \% is the escape character in Lua patterns.
%  
%  \subsection{Examples\label{sec:examples2e}}
%
%  \begin{itemize}
%  \item 
%  Include every |png| file in the current directory.
% \begin{verbatim}
% \QueryFiles{*.png}{\includegraphics{#1}\par}
% \end{verbatim}
%
% \item
%  Input every \TeX\ file with a filename matching \texttt{chapter[0-9].tex}.
% \begin{verbatim}
% \QueryFiles[pattern]{chapter%d.*%.tex}{\input{#1}}
% \end{verbatim}
%  \end{itemize}
%
% \section{The L3 programming layer interface\label{sec:expl3}}
% 
% \begin{function}
%   {\sys_get_query:nN, \sys_get_query:nnN, \sys_get_query:nnnN}
%   \begin{syntax}
%     \cs{sys_get_query:nN} \Arg{cmd} \Arg{tl var}
%     \cs{sys_get_query:nnN} \Arg{cmd} \Arg{spec} \Arg{tl var}
%     \cs{sys_get_query:nnnN} \Arg{cmd} \Arg{options} \Arg{spec} \Arg{tl var}
%   \end{syntax}
%   Sets the \meta{tl var} to the information returned by the
%   \texttt{l3sys-query} \meta{cmd}, potentially supplying the \meta{options}
%   and \meta{spec} to the query call. The valid \meta{cmd} names are at present
%   \begin{itemize}
%     \item \texttt{pwd} Returns the absolute path to the current working directory
%     \item \texttt{ls} Returns a directory listing, using the \meta{spec} to
%       select files and applying the \meta{options} if given
%   \end{itemize}
%   The \meta{spec} should be a file glob and will automatically be passed to
%   the script without shell expansion.
% \end{function}
% 
% \begin{function}
%   {\sys_split_query:nN, \sys_split_query:nnN, \sys_split_query:nnnN}
%   \begin{syntax}
%     \cs{sys_split_query:nN} \Arg{cmd} \Arg{seq}
%     \cs{sys_split_query:nnN} \Arg{cmd} \Arg{spec} \Arg{seq}
%     \cs{sys_split_query:nnnN} \Arg{cmd} \Arg{options} \Arg{spec} \Arg{seq}
%   \end{syntax}
%   Works as described for \cs{sys_get_query:nnnN}, but sets the \meta{seq}
%   to contain one entry for each line returned by \texttt{l3sys-query}.
% \end{function}
% 
% \section{The \texttt{l3sys-query} Lua script\label{sec:lua}}
% \subsection{The command line interface}
% 
% The command line interface to 
% \begin{center}
%   \ttfamily
%   l3sys-query \meta{cmd} [\meta{option(s)}] [\meta{args}]
% \end{center}
% where \texttt{\meta{cmd}} can be one of the following:
% \begin{itemize}[noitemsep]\ttfamily
%   \item ls
%   \item ls \meta{args}
%   \item pwd
% \end{itemize}
% The \meta{cmd} are described below. The result of the \meta{cmd} will be printed
% to the terminal in an interactive run; in normal usage, this will be piped to
% the calling \TeX{} process. Results containing path separators \emph{always}
% use~|/|, irrespective of the platform in use.
% 
% As well as these targets, the script recognizes the options
% \begin{itemize}
%   \item |--exclude| Specification for directory entries to exclude
%   \item |--ignore-case| Ignores case when sorting directory listings
%   \item |--pattern| (|-p|) Treat the \meta{args} as Lua patterns rather
%     than converting from wildcards
%   \item |--recursive| (|-r|) Enables recursive searching during directory
%     listings
%   \item |--reverse| Causes sorting to go from highest to lowest rather
%     than lowest to highest
%   \item |--sort| Sets the method used to sort entries returned by |ls|
%   \item |--type| Selects the type of entry returned by |ls|
% \end{itemize}
% The action of these options on the appropriate \meta{cmd(s)} is detailed below.
% 
% \subsubsection{\texttt{ls [\meta{args}]}}
% 
% Lists the contents of one or more directories, in a manner somewhat reminiscent
% of the Unix command |ls| or the Windows command |dir|. The exact nature of the
% output will depend on the \meta{args}, if given, along with the prevailing
% options. Note that the options names are inspired by ideas from the Unix
% commands |ls| and |find| as well as the Windows command |dir|: they therefore do
% not map directly to those of any one of the command line tools that they
% somewhat mirror.
% 
% When no \meta{args} are given, all entries in the current directory will be
% listed, one per line in the output. This will include both files and
% subdirectories. Each entry will include a path relative to the current
% directory: for files \emph{in} the current directory, this will be |./|. The
% order of results will be determined by the underlying operating system process:
% unless requested \emph{via} an option, no sorting takes place.
% 
% As standard, the \meta{args} are treated as a file/path name potentially
% including |?| and |*| as wildcards, for example |*.png| or |file?.txt|.
% \begin{verbatim}
%   l3sys-query ls '*.png'
% \end{verbatim}
% Some care is needed in preventing expansion of such wildcards by the shell or
% \texttt{texlua} process: these are detailed in Section~\ref{sec:wildcard}. In
% this section, |'| is used to indicate a character being used to suppress
% expansion: this is for example normal on macOS and Linux.
% 
% Removal of entries from the listing can be achieved using the |--exclude|
% option, which should be given with a \meta{xarg}, for example
% \begin{verbatim}
%   l3sys-query ls --exclude '*.bak' 'graphics/*'
% \end{verbatim}
% Directory entries starting |.| are traditionally hidden on Linux and macOS
% systems: these \enquote{dot} entries are excluded from the output of
% \texttt{l3sys-query}. The entries |.| and |..| for the current and parent
% directory are also excluded from the results returned by \texttt{l3sys-query} as
% they can always be assumed.
% 
% For more complex matching, the \meta{args} can be treated as a Lua pattern using
% the |--pattern| (|-p|) option; this also applies to the \meta{xarg} argument to
% the |--exclude| option. For example, the equivalent to wildcard |*.png| could be
% obtained using
% \begin{verbatim}
%   l3sys-query ls --pattern '^.*%.png$'
% \end{verbatim}
% 
% The results returned by |ls| can be sorted using the |--sort| option. This can
% be set to |none| (use the order from the file system: the default), |name| (sort
% by file name) or |date| (sort by date last modified). The sorting order can be
% reversed using |--reverse|. Sorting normally takes account of case: this can be
% suppressed with the |--ignore-case| option.
% 
% The listing can be filtered based on the type of entry using the |--type|
% option. This takes a string argument, one of |d| (directory) or |f| (file).
% 
% As standard, only the path specified as part of the \meta{args} is queried.
% However, if the |--recursive| (|-r|) option is set, the query is applied within
% all subdirectories. Subdirectories starting with~|.| (macOS and Linux hidden)
% are excluded from recursion.
% 
% For security reasons, only paths within the current working directory can be
% queried, thus for example |graphics/*.png| will list all |png| files in the
% |graphics| subdirectory, but |../graphics/*.png| will yield no output.
% 
% \subsubsection{\texttt{pwd}}
% 
% Returns the current working directory from which \texttt{l3sys-query} is run.
% From within a \TeX{} run, this will (usually) be the directory containing the
% main file, assuming a command such as
% \begin{verbatim}
%   pdflatex main.tex
% \end{verbatim}
% The \texttt{pwd} command is unaffected by any options.
% 
% \subsection{Spaces in arguments}
% 
% Since \texttt{l3sys-query} is intended primarily for use with restricted shell
% escape calls from \TeX{} processes, handling of spaces is unusual. It is not
% possible to quote spaces in such a call, so for example whilst
% \begin{verbatim}
%   l3sys-query ls "foo *"
% \end{verbatim}
% does work from the command prompt to find all files with names starting
% \verb*|foo |, it would not work \emph{via} restricted shell escape. To
% circumvent this, \texttt{l3sys-query} will collect all command line arguments
% after any \meta{options}, and combine these as a space-separated \meta{args},
% for example allowing
% \begin{verbatim}
%   l3sys-query ls foo '*'
% \end{verbatim}
% to achieve the same result as the first example. The result is that the
% \meta{args} will only every be interpreted by \texttt{l3sys-query} as a single
% argument. It also means that spaces cannot be used at the start or end of the
% argument, nor can multiple spaces appear between non-space arguments.
% 
% \subsection{Wildcard expansion handling\label{sec:wildcard}}
% 
% The handling of wildcards needs some further comment for those using
% \texttt{l3sys-query} from the command line: the \pkg{expl3} interface described
% in Section~\ref{sec:expl3} handles this aspect automatically for the user.
% 
% On macOS and Linux, the shell normally expands globs, which include the
% wildcards |*| and |?|, before passing arguments to the appropriate command.
% This can be suppressed by surrounding the argument with |'| characters, hence
% the formulation
% \begin{verbatim}
%   l3sys-query ls '*.png'
% \end{verbatim}
% earlier.
% 
% On Windows, the shell does no expansion, and thus arguments are passed as-is to
% the relevant command. As such, |'| has no special meaning here. However, to
% allow quoting of wildcards from the shell in a platform-neutral manner,
% \texttt{l3sys-query} will strip exactly one set of |'| characters around each
% argument before further processing.
% 
% It is not possible to use |"| quotes at all in the argument passed to
% \texttt{l3sys-query} from \TeX{}, as the \TeX{} system removes all |"| in
% |\input| while handling space quoting.
% 
% Restricted shell escape prevents shell expansion of wildcards entirely. On
% non-Windows systems, it does this by ensuring that each argument is |'| quoted
% to ensure further expansion. Thus a \TeX{} call such as
% \begin{verbatim}
%   \input|"l3sys-query ls '*.png'"
% \end{verbatim}
% will work if |--shell-escape| is used as the argument is passed directly to the
% shell, but in restricted shell escape will give an error such as:
% \begin{verbatim}
% ! I can't find file `"|l3sys-query ls '*.png'"'.
% \end{verbatim}
% The \LaTeX{} interfaces described above adjust the quoting used depending on the
% |shell-escape| status.
% 
% 
% \MaybeStop{}
%
% \section{The \LaTeXe\ package implementation}
%
%    \begin{macrocode}
%<*package>
%    \end{macrocode}
%
%    \begin{macrocode}
\ExplSyntaxOn
%<@@=queryfiles>
%    \end{macrocode}
%
% The package should eventually work in restricted shell escape but will do
% nothing useful if the process was started with |--no-shell-escape|.
%    \begin{macrocode}
\sys_if_shell:F{
  \PackageWarningNoLine{l3sys-query}
  {Shell ~Escape ~is ~disabled.\MessageBreak All ~queries ~will ~return ~empty ~results}
  }
%    \end{macrocode}
%
% \subsection{\cs{QueryWorkingDirectory}}
% |\QueryWorkingDirectory| is a direct call to the |pwd| command provided by \texttt{l3sys-query}.
%    \begin{macrocode}
\NewDocumentCommand\QueryWorkingDirectory {m} {
  \sys_get_query:nN {pwd} #1
}
%    \end{macrocode}
%
% \subsection{\cs{QueryFiles} and \cs{QueryFilesTF}}
%
% The declarations of these commands are done in two steps to allow catcode chanes
% before the arguments are read. This allows the use of |%| and |^^| in patterns
% at least if the command is not nested in another command argument. (|\%| may be used
% to generate |%| in all cases).
%
% Variables for saving the current definition of |\%| and |~|.
%    \begin{macrocode}
\tl_new:N\l_query_percent_tl
\tl_new:N\l_query_tilde_tl
%    \end{macrocode}
%
% Allow |%| and |^^| at the top level.
%    \begin{macrocode}
\NewDocumentCommand\QueryFiles {} {
  \group_begin:
    \char_set_catcode_other:N \%
    \char_set_catcode_other:N \^
    \QueryFiles_inner
  }
%    \end{macrocode}
%
%    \begin{macrocode}
\char_set_catcode_active:N \~
%    \end{macrocode}
%
%    \begin{macrocode}
\NewDocumentCommand\QueryFiles_inner {O{}m}{
  \group_end:
  \tl_set:Nn\l_tmpa_tl{}
  \cs_set_eq:NN  \l_query_percent_tl \%
  \cs_set_eq:NN  \% \c_percent_str
  \cs_set_eq:NN  \l_query_tilde_tl ~
  \cs_set_eq:NN  ~ \c_tilde_str
  \keys_set:nn{QueryFiles}{#1}
  \exp_args:NnV\sys_split_query:nnnN {ls} \l_tmpa_tl {#2} \l_tmpa_seq
  \cs_set_eq:NN  \% \l_query_percent_tl
  \cs_set_eq:NN  ~ \l_query_tilde_tl
  \seq_map_inline:Nn\l_tmpa_seq
}
%    \end{macrocode}
%
% This duplicates rather than shares code so as to read the function
% and TF arguments with normal catcode regime.
% (This could probably be optimised.)
%    \begin{macrocode}
\NewDocumentCommand\QueryFilesTF {} {
  \group_begin:
    \char_set_catcode_other:N \%
    \QueryFilesTF_inner
 }
%    \end{macrocode}
%
%    \begin{macrocode}
\NewDocumentCommand\QueryFilesTF_inner {O{}m}{
  \group_end:
  \tl_set:Nn\l_tmpa_tl{}
  \cs_set_eq:NN  \l_query_percent_tl \%
  \cs_set_eq:NN  \% \c_percent_str
  \cs_set_eq:NN  \l_query_tilde_tl ~
  \cs_set_eq:NN  ~ \c_tilde_str
  \keys_set:nn{QueryFiles}{#1}
  \exp_args:NnV\sys_split_query:nnnN {ls} \l_tmpa_tl {#2} \l_tmpa_seq
  \cs_set_eq:NN  \% \l_query_percent_tl
  \cs_set_eq:NN  ~ \l_query_tilde_tl
  \seq_if_empty:NTF \l_tmpa_seq \use_iii:nnn \__queryfiles_aux:nnn
}
%    \end{macrocode}
%
%    \begin{macrocode}
\char_set_catcode_space:N \~
%    \end{macrocode}
%
%    \begin{macrocode}
\cs_new:Npn  \__queryfiles_aux:nnn #1#2#3 {
    #2
    \seq_map_inline:Nn\l_tmpa_seq {#1}
 }
%    \end{macrocode}
%
% Defining the keys. Most take no value and simply add a \texttt{l3sys-query} \texttt{-{}-}
% option to the command linebeing constructed.
%    \begin{macrocode}
\keys_define:nn {QueryFiles} {
%    \end{macrocode}
%
%    \begin{macrocode}
recursive .code:n  =\tl_put_right:Nn \l_tmpa_tl {--recursive ~ } ,
recursive .value_forbidden:n = true ,
%    \end{macrocode}
%
%    \begin{macrocode}
ignore-case .code:n  =\tl_put_right:Nn \l_tmpa_tl {--ignore-case ~ } ,
ignore-case .value_forbidden:n = true ,
%    \end{macrocode}
%
%    \begin{macrocode}
reverse .code:n  =\tl_put_right:Nn \l_tmpa_tl {--reverse ~ } ,
reverse .value_forbidden:n = true ,
%    \end{macrocode}
%
%    \begin{macrocode}
exclude .code:n  =\tl_put_right:Ne \l_tmpa_tl {
  --exclude ~
  \sys_if_shell_restricted:F'
  \exp_not:n{#1}
  \sys_if_shell_restricted:F'
  ~ } ,
exclude .value_required:n = true ,
%    \end{macrocode}
%
% The |type| key checks the supplied value is valid |d| for directories
% or |f| for files The default behaviour lists both.
%    \begin{macrocode}
type .choices:nn = {d,f}
                   {\tl_put_right:Nn \l_tmpa_tl {--type ~ #1 ~ }} ,
%    \end{macrocode}
%
% The |sort| key checks the supplied value is valid |date| or |name|.
%    \begin{macrocode}
sort .choices:nn = {date,name}
{\tl_put_right:Nn \l_tmpa_tl {--sort ~ #1 ~ }} ,
%    \end{macrocode}
%
%    \begin{macrocode}
pattern .code:n  =\tl_put_right:Nn \l_tmpa_tl {--pattern ~ } ,
pattern .value_forbidden:n = true ,
%    \end{macrocode}
%
%    \begin{macrocode}
}
%    \end{macrocode}
%
%    \begin{macrocode}
\ExplSyntaxOff
%    \end{macrocode}
%
%    \begin{macrocode}
%</package>
%    \end{macrocode}
%
% \Finale
%
