% \iffalse meta-comment
%
% Copyright (C) 1993-2024
%
% The LaTeX Project and any individual authors listed elsewhere
% in this file.
%
% This file is part of the Standard LaTeX `Tools Bundle'.
% -------------------------------------------------------
%
% It may be distributed and/or modified under the
% conditions of the LaTeX Project Public License, either version 1.3c
% of this license or (at your option) any later version.
% The latest version of this license is in
%    https://www.latex-project.org/lppl.txt
% and version 1.3c or later is part of all distributions of LaTeX
% version 2005/12/01 or later.
%
% The list of all files belonging to the LaTeX `Tools Bundle' is
% given in the file `manifest.txt'.
%
% \fi
% \iffalse
%% File: xr.dtx Copyright (C) 1993-2019 David Carlisle
%
%<package>\NeedsTeXFormat{LaTeX2e}
%<package>\ProvidesPackage{xr}
%<package>         [2023-07-04 v5.06 eXternal References (DPC)]
%
%<*driver>
\documentclass{ltxdoc}
\usepackage{xr}
\GetFileInfo{xr.sty}
\begin{document}
\title{The \textsf{xr} package\thanks{This file
        has version number \fileversion, last
        revised \filedate.}}
\author{David Carlisle\thanks{%
  The Author of Versions 1--4 was Jean-Pierre Drucbert}}
\date{\filedate}
\MaintainedByLaTeXTeam{tools}
\maketitle
\DocInput{xr.dtx}
\end{document}
%</driver>
% \fi
%
% %%%%%%%%%%%%%%%%%%%%%%%%%%%%%%%%%%%%%%%%%%%%%%%%%%%%%%%%%%%%%%%%%%%%
%
%
% \changes{v5.00}{1993/07/07}
%         {First DPC version (by agreement with J-PD).  New mechanism
%         (\cmd{\read} instead of \cmd\input).}
%
% \changes{v5.01}{1993/07/20}{Fix bug added by DPC, v5.00 did not import
%           aux files of \cmd\include'ed files. (Reported by J-PD)}
%
% \changes{v5.02}{1994/05/28}{Update for LaTeX2e}
% \changes{v5.03}{2018/10/01}{Fix for conditionals in aux file}
% \changes{v5.05}{2019/07/20}{include xcite}
%
%
% This package implements a system for eXternal References.
%
% If one document needs to refer to sections of another, say |aaa.tex|,
% then this package may be loaded in the main file, and the command\\
% |\externaldocument{aaa}|\\
%  given in the preamble.
%
% Then you may use |\ref| and |\pageref| to refer to anything which has
% been given a |\label| in either |aaa.tex| or the main document.
% You may declare any number of such external documents.
%
% If any of the external documents, or the main document, use the same
% |\label| then an error will occur as the label will be multiply
% defined. To overcome this problem |\externaldocument| has an optional
% argument. If you declare |\externaldocument[A-]{aaa}|, then all
% references from |aaa| are prefixed by |A-|. So for instance, if a
% section of |aaa| had |\label{intro}|, then this could be referenced
% with |\ref{A-intro}|. The prefix need not be |A-|, it can be any
% string chosen to ensure that all the labels imported from external
% files are unique. Note however that if your style declares certain
% active characters (|:| in French, |"| in German) then these
% characters can not usually be used in |\label|, and similarly may not
% be used in the optional argument to |\externaldocument|.
%
% As first suggested in Enrico Gregorio's |xcite| package, the current version
% also allows |\cite| to reference |\bibitem| in the external document.
% For compatibility with |xcite|, |\externalcitedocument| is made available
% as an alias for |\externaldocument|
%
% \MaybeStop{}
%
% \section{The macros}
%
%    \begin{macrocode}
%<*package>
%    \end{macrocode}
%
% Check for the optional argument.
%    \begin{macrocode}
\def\externaldocument{\@ifnextchar[\XR@{\XR@[]}}
\let\externalcitedocument\externaldocument
%    \end{macrocode}
%
%    Save the optional prefix. Start processing the first |aux| file.
%  \changes{v5.06}{2020-05-10}{Remove leading and trailing sapces from
%    the filename (gh/2223)}
%    \begin{macrocode}
\def\XR@[#1]#2{{%
  \makeatletter
  \def\XR@prefix{#1}%
  \set@curr@file{#2}%
  \expandafter\XR@next\@curr@file.aux\relax\\}}
%    \end{macrocode}
%
% Process the next |aux| file in the list and remove it from the head of
% the list of files to process.
%    \begin{macrocode}
\def\XR@next#1\relax#2\\{%
  \edef\XR@list{#2}%
  \XR@loop{#1}}
%    \end{macrocode}
%
% Check whether the list of |aux| files is empty.
%    \begin{macrocode}
\def\XR@aux{%
  \ifx\XR@list\@empty\else\expandafter\XR@explist\fi}
%    \end{macrocode}
%

% Expand the list of aux files, and call |\XR@next| to process the first
% one.
%    \begin{macrocode}
\def\XR@explist{\expandafter\XR@next\XR@list\\}
%    \end{macrocode}
%
% If the |aux| file exists, loop through line by line, looking for
% |\newlabel| and |\@input|. Otherwise process the next file in the
% list.
%  \changes{v5.06}{2020-05-10}{Add braces around the filename to
%    support filenames with spaces (gh/223)}
%    \begin{macrocode}
\def\XR@loop#1{\openin\@inputcheck{#1}\relax
  \ifeof\@inputcheck
    \PackageWarning{xr}{^^JNo file #1^^JLABELS NOT IMPORTED.^^J}%
    \expandafter\XR@aux
  \else
    \PackageInfo{xr}{IMPORTING LABELS FROM #1}%
    \expandafter\XR@read\fi}
%    \end{macrocode}
%
% Read the next line of the aux file.
%    \begin{macrocode}
\def\XR@read{%
  \read\@inputcheck to\XR@line
%    \end{macrocode}
% The |...| make sure that |\XR@test| always has sufficient arguments.
%    \begin{macrocode}
  \expandafter\XR@test\XR@line...\XR@}
%    \end{macrocode}
%
% Look at the first token of the line.
% If it is |\newlabel|, do the |\newlabel|. If it is |\@input|, add the
% filename to the list of files to process. Otherwise ignore.
% Go around the loop if not at end of file. Finally process the next
% file in the list.
%
% 2018 update: make sure the arguments are handled outside the |\ifx| test,
%    \begin{macrocode}
\long\def\XR@test#1#2#3#4\XR@{%
  \let\XR@tempa\@gobbletwo
  \ifx#1\newlabel
    \let\XR@tempa\@firstoftwo
  \else\ifx#1\bibcite
    \let\XR@tempa\@firstoftwo
  \else\ifx#1\@input
     \let\XR@tempa\@secondoftwo
  \fi\fi\fi
   \XR@tempa{#1{\XR@prefix#2}{#3}}{\edef\XR@list{\XR@list#2\relax}}%
  \ifeof\@inputcheck\expandafter\XR@aux
  \else\expandafter\XR@read\fi}
%    \end{macrocode}
%
%    \begin{macrocode}
%</package>
%    \end{macrocode}
%
% \Finale
%
