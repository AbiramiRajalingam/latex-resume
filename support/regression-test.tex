%%
%% This is file `regression-test.tex',
%% generated with the docstrip utility.
%%
%% The original source files were:
%%
%% l3build.dtx  (with options: `package')
%% 
%% Copyright (C) 2014-2021 The LaTeX Project
%% 
%% It may be distributed and/or modified under the conditions of
%% the LaTeX Project Public License (LPPL), either version 1.3c of
%% this license or (at your option) any later version.  The latest
%% version of this license is in the file:
%% 
%%    http://www.latex-project.org/lppl.txt
%% 
%% This file is part of the "l3build bundle" (The Work in LPPL)
%% and all files in that bundle must be distributed together.
%% 
\begingroup\expandafter\expandafter\expandafter\endgroup
\expandafter\ifx\csname eTeXversion\endcsname\relax
  \message{e-TeX is required by some regression-test.tex features}%
\fi
\ifx\unprotect\undefined
  \expandafter\edef\csname reset\string @catcodes\endcsname{%
    \catcode`\noexpand\@=\the\catcode`\@\relax
  }%
  \catcode`\@=11 %
\else
  \unprotect
  \def\reset@catcodes{\protect}%
\fi
\begingroup\expandafter\expandafter\expandafter\endgroup
\expandafter\ifx\csname interactionmode\endcsname\relax
\else
  \ifnum\interactionmode>1 \scrollmode\fi
\fi
\errorcontextlines=-1 %
\showboxbreadth=\maxdimen
\showboxdepth=\maxdimen
\def\loggingoutput{%
  \tracingoutput=1 %
  \showboxbreadth=\maxdimen
  \showboxdepth=\maxdimen
}
\newlinechar=`\^^J
\long\def\LONGTYPEOUT#1{%
  \begingroup
    \long\def\TYPE##1{##1}%
    \immediate\write128{#1}%
  \endgroup
}
\let\TYPE\LONGTYPEOUT
\def\STARTMESSAGE{This is a generated file for the l3build validation system.}
\def\START{%
  \LONGTYPEOUT{^^JSTART-TEST-LOG^^J}%
  \LONGTYPEOUT{^^J%
     \STARTMESSAGE%
     ^^J^^JDon't change this file in any respect.%
     ^^J^^J%
  }%
}
\ifx\@@end\@undefined
  \let\@@@end\end
\else
  \let\@@@end\@@end
\fi
\def\END{%
  \ifnum\currentgrouplevel>0 %
    \LONGTYPEOUT{Bad grouping: \the\currentgrouplevel!}%
  \fi
  \ifnum\currentiflevel>2 %
    \LONGTYPEOUT{Bad conditionals: \the\numexpr\currentiflevel-2!}%
  \fi
  \LONGTYPEOUT{^^JEND-TEST-LOG^^J}%
  \@@@end
}
\begingroup\expandafter\expandafter\expandafter\endgroup
\expandafter\ifx\csname currentgrouplevel\endcsname\relax
  \def\END{%
    \LONGTYPEOUT{^^JEND-TEST-LOG^^J}%
    \@@@end
  }
\fi
\ifx\@@end\@undefined
  \let\end\END
\else
  \let\@@end\END
\fi
\def\OMIT{\LONGTYPEOUT{OMIT}}
\def\TIMO{\LONGTYPEOUT{TIMO}}
\begingroup       % within the scope of this groups each line needs to end in % !
\catcode`\^^M\active %
\gdef\SHOWFILE#1{%
     \typeout{-------- #1 (start) ---------}%
\IfFileExists{#1}%
  {\begingroup %
     \catcode`\^^M\active %
     \edef^^M{^\string^M^^J}%
     \everyeof{\noexpand}%
     \obeyspaces %
     \@sanitize %
     \message{\@@input #1 }%
   \endgroup }%
  {\message{Not found}}%
     \typeout{-------- #1 (end) -----------}%
}%
\endgroup
\begingroup\expandafter\expandafter\expandafter\endgroup
\expandafter\ifx\csname everyeof\endcsname\relax
  \def\SHOWFILE#1{\TYPE{FEATURE UNAVAILABLE}}
\fi
\ifx\InputIfFileExists\@undefined
  \newread\@inputcheck
  \long\def\InputIfFileExists#1#2#3{%
    \openin\@inputcheck#1\relax
    \ifeof\@inputcheck
      \def\reserved@a{#3}%
    \else
      \def\reserved@a{#2\input #1\relax}%
    \fi
    \closein\@inputcheck
    \reserved@a
  }%
\fi
\InputIfFileExists{regression-test.cfg}
  {\LONGTYPEOUT{^^J***^^Jregression-test.cfg in operation^^J***^^J}}{}
\newcount\gTESTint
\def\SEPARATOR{%
  \TYPE{%
    ============================================================%
  }%
}
\csname protected\endcsname\long\def\BEGINTEST#1{%
  \global\advance\gTESTint by 1 %
  \SEPARATOR
  \LONGTYPEOUT{TEST \the\gTESTint: \detokenize{#1}}%
  \SEPARATOR
  \begingroup
    \let\TYPE\LONGTYPEOUT
}
\begingroup\expandafter\expandafter\expandafter\endgroup
\expandafter\ifx\csname detokenize\endcsname\relax
  \long\def\BEGINTEST#1{%
    \global\advance\gTESTint by 1 %
    \SEPARATOR
    \begingroup
      \toks0={#1}%
      \LONGTYPEOUT{TEST \the\gTESTint: \the\toks0 }%
    \endgroup
    \SEPARATOR
    \begingroup
      \let\TYPE\LONGTYPEOUT
  }
\fi
\csname protected\endcsname\long\def\ENDTEST{%
  \endgroup
  \SEPARATOR
  \LONGTYPEOUT{}%
}
\long\def\TEST#1#2{%
  \BEGINTEST{#1}%
  #2%
  \ENDTEST
}
\long\def\TESTEXP#1#2{%
  \BEGINTEST{#1}%
  \LONGTYPEOUT{#2}%
  \ENDTEST
}
\def \TRUE  {\TYPE{TRUE}}
\def \FALSE {\TYPE{FALSE}}
\def \YES   {\TYPE{YES}}
\def \NO    {\TYPE{NO}}
\def \NEWLINE {\TYPE{^^J}}
\csname protected\endcsname\long\def\ASSERT#1#2{%
  \begingroup
    \edef\@tempa{#1}%
    \edef\@tempb{#2}%
    \ifx\@tempa\@tempb
      \TYPE{PASSED}%
    \else
      \TYPE{FAILED}%
    \fi
  \endgroup
}
\csname protected\endcsname\long\def\ASSERTSTR#1#2{%
  \begingroup
    \edef\@tempa{#1}%
    \edef\@tempb{#2}%
    \edef\@tempa{\meaning\@tempa}%
    \edef\@tempb{\meaning\@tempb}%
    \ifx\@tempa\@tempb
      \TYPE{PASSED}%
    \else
      \TYPE{FAILED}%
    \fi
  \endgroup
}
\begingroup\expandafter\expandafter\expandafter\endgroup
\expandafter\ifx\csname detokenize\endcsname\relax
  \long\def\ASSERTSTR#1#2{%
    \TYPE{FEATURE UNAVAILABLE}%
  }
\fi
\ifnum 0%
  \ifx\pdfoutput\@undefined\else\ifnum\pdfoutput>0 1\fi\fi
  \ifx\outputmode\@undefined\else\ifnum\outputmode>0 1\fi\fi
  >0 %
  \ifx\pdfvariable\@undefined
    \pdfcompresslevel=0 %
    \pdfobjcompresslevel=0 %
  \else
    \pdfvariable compresslevel=0 %
    \pdfvariable objcompresslevel=0 %
  \fi
\else
  \ifx\XeTeXversion\@undefined
  \special{%
      ps: /setdistillerparams
      where
        {pop << /CompressPages false /CompressStreams false >> setdistillerparams}
      if
    }%
  \else
    \special{dvipdfmx:config z 0}% Compress level
    \special{dvipdfmx:config C 0x40}% Object compression
  \fi
\fi
\begingroup\expandafter\expandafter\expandafter\endgroup
\expandafter\ifx\csname protected\endcsname\relax
  \reset@catcodes
  \let\protected\undefined
  \expandafter\endinput
\fi
\ifx\pdfoutput\@undefined
  \ifx\outputmode\@undefined
  \else
    \ifnum\outputmode>0 %
      \pdfextension mapfile{pdftex.map}%
    \fi
  \fi
\else
  \ifnum\pdfoutput>0 %
    \pdfmapfile{pdftex.map}%
  \fi
\fi
\ifcsname pdfmeta_set_regression_data:\endcsname
   \csname pdfmeta_set_regression_data:\endcsname
\else
  \ifnum 0%
    \ifx\pdfoutput\@undefined\else\ifnum\pdfoutput>0 1\fi\fi
    \ifx\outputmode\@undefined\else\ifnum\outputmode>0 1\fi\fi
    >0 %
    \ifx\pdfvariable\@undefined
      \pdfinfo{/Producer (\ifx\directlua\@undefined pdf\else Lua\fi TeX)}
      \ifx\pdfinfoomitdate\@undefined\else
        \pdfinfoomitdate     = 1 %
        \pdfsuppressptexinfo = \numexpr
              0
            + 1 % PTEX.Fullbanner
            + 2 % PTEX.FileName
          \relax
        \pdftrailerid{}
      \fi
    \else
      \pdfextension info{/Producer (LuaTeX)}
      \pdfvariable suppressoptionalinfo \numexpr
            0
          +   1 % PTEX.Fullbanner
          +   2 % PTEX.FileName
          +  32 % CreationDate
          +  64 % ModDate
          + 512 % ID
        \relax
    \fi
  \else
    \ifx\XeTeXversion\@undefined
      \special{! <</DocumentUUID (DocumentUUID)>> setpagedevice}
      \special{! <</InstanceUUID (InstanceUUID)>> setpagedevice}
    \else
      \special{%
        pdf: docinfo
          <<
            /Creator        (TeX)
            /CreationDate   ()
            /ModDate        ()
            /Producer       (xdvipdfmx)
          >>
      }
    \fi
  \fi
\fi
\reset@catcodes
%%

%%% this needs to be moved to the source of regression-test.tex!

\ifx\AddToHook\undefined
\else

%
% We don't want release information showing up in \end{document}
% because that would change over and over again.
% We therefore drop that information from the hook (should also
% happen in the standard regression test when running a LaTeX
% format). By using a voids relation we can reactivate the code in
% a test by dropping "the testmode" label.
%
\AddToHook{enddocument/info}[kernel/testmode]{}
\DeclareHookRule{enddocument/info}{kernel/testmode}{voids}{kernel/release}

\fi

%%
%% End of file `regression-test.tex'.

%%%%%%%%%% ADDENDUM by JL
% Adding some assertions
% Do not move this to official regression-test.tex
% because it actually creates tests that are tightly tied to
% the implementation details.
% However, this can live in a separate file, which name is to be found,
% and included by a '*.lvt' file only on demand.

\ExplSyntaxOn

\tl_new:N \l__TEST_lthooks_print_tl
\cs_new:Npn \TESTIfMatchTermOutputTF #1 #2
% #1 is a regular expression,
% #2 is a standalone expression
% Next arguments are the usual T and F arguments
% #2 is executed, term output is catched, and tested against #1
% then branch occurs. 
  {
    \group_begin:
    \tl_clear:N \l__TEST_lthooks_print_tl
    \cs_set:Npn \iow_term:n
      {
        \tl_if_empty:NF  \l__TEST_lthooks_print_tl
          { \tl_put_right:Nn \l__TEST_lthooks_print_tl { ^^J } }
        \tl_put_right:Nn \l__TEST_lthooks_print_tl
      }
% Do not use \cs_generate_variant:Nn to have full control on \iow_term:x
    \cs_set:Npn \iow_term:x { \exp_args:Nx \iow_term:n }
    \cs_set_eq:NN \iow_log:n \iow_term:n
    \cs_set_eq:NN \iow_log:x \iow_term:x
    \cs_set:Npn \tl_show:n ##1 { \iow_term:n { ##1. } }
    \cs_set:Npn \tl_show:x { \exp_args:Nx \tl_show:n }
    \cs_set_eq:NN \tl_log:n \tl_show:n 
    \cs_set_eq:NN \tl_log:x \tl_show:x
    \cs_set:Npn \TYPE { \exp_args:Ne \iow_term:n }
    #2
    \exp_args:NNnV
    \group_end:
    \regex_match:nnTF { #1 } \l__TEST_lthooks_print_tl
% T and F arguments are expected
  }
\cs_new:Npn \ASSERTMatchDebugPrintHook #1 #2
% #1 is a regular expression
% #2 is a hook specification
% Catches \DebugPrintHook term output and tests again the regular expression
% Prints "PASSED" or "FAILED" accordingly
  {
    \TESTIfMatchTermOutputTF { #1 } { \DebugPrintHook { #2 } }
      { \TYPE { PASSED } }
      { \TYPE { FAILED~(NO~MATCH:~\tl_to_str:n { #1 }) } }
  }

\cs_new:Npn \ASSERTUnmatchDebugPrintHook #1 #2
% #1 is a regular expression
% #2 is a hook specification
% Catches \DebugPrintHook term output and tests again the regular expression
% Prints "PASSED" or "FAILED" accordingly
  {
    \TESTIfMatchTermOutputTF { #1 } { \DebugPrintHook { #2 } }
      { \TYPE { FAILED~(MATCH:~\tl_to_str:n { #1 }) } }
      { \TYPE { PASSED } }
  }

\cs_new:Npn \ASSERTMatchShowHook #1 #2
% #1 is a regular expression
% #2 is a hook specification
% Catches \ShowHook term output and tests again the regular expression
% Prints "PASSED" or "FAILED" accordingly
% Beware that the output includes extra information due to the
% user interaction request.
  {
    \TESTIfMatchTermOutputTF { #1 } { \ShowHook { #2 } }
      { \TYPE { PASSED } }
      { \TYPE { FAILED~(NO~MATCH:~\tl_to_str:n { #1 }) } }
  }

\cs_new:Npn \ASSERTUnmatchShowHook #1 #2
% #1 is a regular expression
% #2 is a hook specification
% Catches \ShowHook term output and tests again the regular expression
% Prints "PASSED" or "FAILED" accordingly
% Beware that the output includes extra information due to the
% user interaction request.
  {
    \TESTIfMatchTermOutputTF { #1 } { \ShowHook { #2 } }
      { \TYPE { FAILED~(MATCH:~\tl_to_str:n { #1 }) } }
      { \TYPE { PASSED } }
  }

\ExplSyntaxOff
